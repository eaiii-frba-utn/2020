%% Generated by Sphinx.
\def\sphinxdocclass{report}
\documentclass[a4paper,10pt,spanish]{report}
\ifdefined\pdfpxdimen
   \let\sphinxpxdimen\pdfpxdimen\else\newdimen\sphinxpxdimen
\fi \sphinxpxdimen=.75bp\relax

\PassOptionsToPackage{warn}{textcomp}
\usepackage[utf8]{inputenc}
\ifdefined\DeclareUnicodeCharacter
% support both utf8 and utf8x syntaxes
  \ifdefined\DeclareUnicodeCharacterAsOptional
    \def\sphinxDUC#1{\DeclareUnicodeCharacter{"#1}}
  \else
    \let\sphinxDUC\DeclareUnicodeCharacter
  \fi
  \sphinxDUC{00A0}{\nobreakspace}
  \sphinxDUC{2500}{\sphinxunichar{2500}}
  \sphinxDUC{2502}{\sphinxunichar{2502}}
  \sphinxDUC{2514}{\sphinxunichar{2514}}
  \sphinxDUC{251C}{\sphinxunichar{251C}}
  \sphinxDUC{2572}{\textbackslash}
\fi
\usepackage{cmap}
\usepackage[T1]{fontenc}
\usepackage{amsmath,amssymb,amstext}
\usepackage{babel}


\usepackage{amsmath,amsfonts,amssymb,amsthm}

\usepackage{fncychap}
\usepackage{sphinx}
\sphinxsetup{hmargin={0.7in,0.7in}, vmargin={1in,1in},         verbatimwithframe=true,         TitleColor={rgb}{0,0,0},         HeaderFamily=\rmfamily\bfseries,         InnerLinkColor={rgb}{0,0,1},         OuterLinkColor={rgb}{0,0,1}}
\fvset{fontsize=\small}
\usepackage{geometry}


% Include hyperref last.
\usepackage{hyperref}
% Fix anchor placement for figures with captions.
\usepackage{hypcap}% it must be loaded after hyperref.
% Set up styles of URL: it should be placed after hyperref.
\urlstyle{same}
\addto\captionsspanish{\renewcommand{\contentsname}{Introducción}}

\usepackage{sphinxmessages}
\setcounter{tocdepth}{2}


% Jupyter Notebook code cell colors
\definecolor{nbsphinxin}{HTML}{307FC1}
\definecolor{nbsphinxout}{HTML}{BF5B3D}
\definecolor{nbsphinx-code-bg}{HTML}{F5F5F5}
\definecolor{nbsphinx-code-border}{HTML}{E0E0E0}
\definecolor{nbsphinx-stderr}{HTML}{FFDDDD}
% ANSI colors for output streams and traceback highlighting
\definecolor{ansi-black}{HTML}{3E424D}
\definecolor{ansi-black-intense}{HTML}{282C36}
\definecolor{ansi-red}{HTML}{E75C58}
\definecolor{ansi-red-intense}{HTML}{B22B31}
\definecolor{ansi-green}{HTML}{00A250}
\definecolor{ansi-green-intense}{HTML}{007427}
\definecolor{ansi-yellow}{HTML}{DDB62B}
\definecolor{ansi-yellow-intense}{HTML}{B27D12}
\definecolor{ansi-blue}{HTML}{208FFB}
\definecolor{ansi-blue-intense}{HTML}{0065CA}
\definecolor{ansi-magenta}{HTML}{D160C4}
\definecolor{ansi-magenta-intense}{HTML}{A03196}
\definecolor{ansi-cyan}{HTML}{60C6C8}
\definecolor{ansi-cyan-intense}{HTML}{258F8F}
\definecolor{ansi-white}{HTML}{C5C1B4}
\definecolor{ansi-white-intense}{HTML}{A1A6B2}
\definecolor{ansi-default-inverse-fg}{HTML}{FFFFFF}
\definecolor{ansi-default-inverse-bg}{HTML}{000000}

% Define an environment for non-plain-text code cell outputs (e.g. images)
\makeatletter
\newenvironment{nbsphinxfancyoutput}{%
    % Avoid fatal error with framed.sty if graphics too long to fit on one page
    \let\sphinxincludegraphics\nbsphinxincludegraphics
    \nbsphinx@image@maxheight\textheight
    \advance\nbsphinx@image@maxheight -2\fboxsep   % default \fboxsep 3pt
    \advance\nbsphinx@image@maxheight -2\fboxrule  % default \fboxrule 0.4pt
    \advance\nbsphinx@image@maxheight -\baselineskip
\def\nbsphinxfcolorbox{\spx@fcolorbox{nbsphinx-code-border}{white}}%
\def\FrameCommand{\nbsphinxfcolorbox\nbsphinxfancyaddprompt\@empty}%
\def\FirstFrameCommand{\nbsphinxfcolorbox\nbsphinxfancyaddprompt\sphinxVerbatim@Continues}%
\def\MidFrameCommand{\nbsphinxfcolorbox\sphinxVerbatim@Continued\sphinxVerbatim@Continues}%
\def\LastFrameCommand{\nbsphinxfcolorbox\sphinxVerbatim@Continued\@empty}%
\MakeFramed{\advance\hsize-\width\@totalleftmargin\z@\linewidth\hsize\@setminipage}%
\lineskip=1ex\lineskiplimit=1ex\raggedright%
}{\par\unskip\@minipagefalse\endMakeFramed}
\makeatother
\newbox\nbsphinxpromptbox
\def\nbsphinxfancyaddprompt{\ifvoid\nbsphinxpromptbox\else
    \kern\fboxrule\kern\fboxsep
    \copy\nbsphinxpromptbox
    \kern-\ht\nbsphinxpromptbox\kern-\dp\nbsphinxpromptbox
    \kern-\fboxsep\kern-\fboxrule\nointerlineskip
    \fi}
\newlength\nbsphinxcodecellspacing
\setlength{\nbsphinxcodecellspacing}{0pt}

% Define support macros for attaching opening and closing lines to notebooks
\newsavebox\nbsphinxbox
\makeatletter
\newcommand{\nbsphinxstartnotebook}[1]{%
    \par
    % measure needed space
    \setbox\nbsphinxbox\vtop{{#1\par}}
    % reserve some space at bottom of page, else start new page
    \needspace{\dimexpr2.5\baselineskip+\ht\nbsphinxbox+\dp\nbsphinxbox}
    % mimick vertical spacing from \section command
      \addpenalty\@secpenalty
      \@tempskipa 3.5ex \@plus 1ex \@minus .2ex\relax
      \addvspace\@tempskipa
      {\Large\@tempskipa\baselineskip
             \advance\@tempskipa-\prevdepth
             \advance\@tempskipa-\ht\nbsphinxbox
             \ifdim\@tempskipa>\z@
               \vskip \@tempskipa
             \fi}
    \unvbox\nbsphinxbox
    % if notebook starts with a \section, prevent it from adding extra space
    \@nobreaktrue\everypar{\@nobreakfalse\everypar{}}%
    % compensate the parskip which will get inserted by next paragraph
    \nobreak\vskip-\parskip
    % do not break here
    \nobreak
}% end of \nbsphinxstartnotebook

\newcommand{\nbsphinxstopnotebook}[1]{%
    \par
    % measure needed space
    \setbox\nbsphinxbox\vbox{{#1\par}}
    \nobreak % it updates page totals
    \dimen@\pagegoal
    \advance\dimen@-\pagetotal \advance\dimen@-\pagedepth
    \advance\dimen@-\ht\nbsphinxbox \advance\dimen@-\dp\nbsphinxbox
    \ifdim\dimen@<\z@
      % little space left
      \unvbox\nbsphinxbox
      \kern-.8\baselineskip
      \nobreak\vskip\z@\@plus1fil
      \penalty100
      \vskip\z@\@plus-1fil
      \kern.8\baselineskip
    \else
      \unvbox\nbsphinxbox
    \fi
}% end of \nbsphinxstopnotebook

% Ensure height of an included graphics fits in nbsphinxfancyoutput frame
\newdimen\nbsphinx@image@maxheight % set in nbsphinxfancyoutput environment
\newcommand*{\nbsphinxincludegraphics}[2][]{%
    \gdef\spx@includegraphics@options{#1}%
    \setbox\spx@image@box\hbox{\includegraphics[#1,draft]{#2}}%
    \in@false
    \ifdim \wd\spx@image@box>\linewidth
      \g@addto@macro\spx@includegraphics@options{,width=\linewidth}%
      \in@true
    \fi
    % no rotation, no need to worry about depth
    \ifdim \ht\spx@image@box>\nbsphinx@image@maxheight
      \g@addto@macro\spx@includegraphics@options{,height=\nbsphinx@image@maxheight}%
      \in@true
    \fi
    \ifin@
      \g@addto@macro\spx@includegraphics@options{,keepaspectratio}%
    \fi
    \setbox\spx@image@box\box\voidb@x % clear memory
    \expandafter\includegraphics\expandafter[\spx@includegraphics@options]{#2}%
}% end of "\MakeFrame"-safe variant of \sphinxincludegraphics
\makeatother

\makeatletter
\renewcommand*\sphinx@verbatim@nolig@list{\do\'\do\`}
\begingroup
\catcode`'=\active
\let\nbsphinx@noligs\@noligs
\g@addto@macro\nbsphinx@noligs{\let'\PYGZsq}
\endgroup
\makeatother
\renewcommand*\sphinxbreaksbeforeactivelist{\do\<\do\"\do\'}
\renewcommand*\sphinxbreaksafteractivelist{\do\.\do\,\do\:\do\;\do\?\do\!\do\/\do\>\do\-}
\makeatletter
\fvset{codes*=\sphinxbreaksattexescapedchars\do\^\^\let\@noligs\nbsphinx@noligs}
\makeatother


        %%%%%%%%%%%%%%%%%%%% Meher %%%%%%%%%%%%%%%%%%
        %%%add number to subsubsection 2=subsection, 3=subsubsection
        %%% below subsubsection is not good idea.
        \setcounter{secnumdepth}{3}
        %
        %%%% Table of content upto 2=subsection, 3=subsubsection
        \setcounter{tocdepth}{2}

        \usepackage{amsmath,amsfonts,amssymb,amsthm}
        \usepackage{graphicx}

        %%% reduce spaces for Table of contents, figures and tables
        %%% it is used "\addtocontents{toc}{\vskip -1.2cm}" etc. in the document
        \usepackage[notlot,nottoc,notlof]{}

        \usepackage{color}
        \usepackage{transparent}
        \usepackage{eso-pic}
        \usepackage{lipsum}

        \usepackage{footnotebackref} %%link at the footnote to go to the place of footnote in the text

        %% spacing between line
        \usepackage{setspace}
        %%%%\onehalfspacing
        %%%%\doublespacing
        \singlespacing


        %%%%%%%%%%% datetime
        \usepackage{datetime}

        \newdateformat{MonthYearFormat}{%
            \monthname[\THEMONTH], \THEYEAR}


        %% RO, LE will not work for 'oneside' layout.
        %% Change oneside to twoside in document class
        \usepackage{fancyhdr}
        \pagestyle{fancy}
        \fancyhf{}

        %%% Alternating Header for oneside
        \fancyhead[L]{\ifthenelse{\isodd{\value{page}}}{ \small \nouppercase{\leftmark} }{}}
        \fancyhead[R]{\ifthenelse{\isodd{\value{page}}}{}{ \small \nouppercase{\rightmark} }}

        %%% Alternating Header for two side
        %\fancyhead[RO]{\small \nouppercase{\rightmark}}
        %\fancyhead[LE]{\small \nouppercase{\leftmark}}

        %% for oneside: change footer at right side. If you want to use Left and right then use same as header defined above.
        \fancyfoot[R]{\ifthenelse{\isodd{\value{page}}}{{\tiny Meher Krishna Patel} }{\href{http://pythondsp.readthedocs.io/en/latest/pythondsp/toc.html}{\tiny PythonDSP}}}

        %%% Alternating Footer for two side
        %\fancyfoot[RO, RE]{\scriptsize Meher Krishna Patel (mekrip@gmail.com)}

        %%% page number
        \fancyfoot[CO, CE]{\thepage}

        \renewcommand{\headrulewidth}{0.5pt}
        \renewcommand{\footrulewidth}{0.5pt}

        \RequirePackage{tocbibind} %%% comment this to remove page number for following
        \addto\captionsenglish{\renewcommand{\contentsname}{Table of contents}}
        \addto\captionsenglish{\renewcommand{\listfigurename}{List of figures}}
        \addto\captionsenglish{\renewcommand{\listtablename}{List of tables}}
        % \addto\captionsenglish{\renewcommand{\chaptername}{Chapter}}


        %%reduce spacing for itemize
        \usepackage{enumitem}
        \setlist{nosep}

        %%%%%%%%%%% Quote Styles at the top of chapter
        \usepackage{epigraph}
        \setlength{\epigraphwidth}{0.8\columnwidth}
        \newcommand{\chapterquote}[2]{\epigraphhead[60]{\epigraph{\textit{#1}}{\textbf {\textit{--#2}}}}}
        %%%%%%%%%%% Quote for all places except Chapter
        \newcommand{\sectionquote}[2]{{\quote{\textit{``#1''}}{\textbf {\textit{--#2}}}}}
    

\title{Sphinx format for Latex and HTML}
\date{04 de mayo de 2020}
\release{}
\author{Meher Krishna Patel}
\newcommand{\sphinxlogo}{\vbox{}}
\renewcommand{\releasename}{ }
\makeindex
\begin{document}

\ifdefined\shorthandoff
  \ifnum\catcode`\=\string=\active\shorthandoff{=}\fi
  \ifnum\catcode`\"=\active\shorthandoff{"}\fi
\fi

\pagestyle{empty}

        \pagenumbering{Roman} %%% to avoid page 1 conflict with actual page 1

        \begin{titlepage}
            \centering

            \vspace*{40mm} %%% * is used to give space from top
            \textbf{\Huge {Sphinx format for Latex and HTML}}

            \vspace{0mm}
            \begin{figure}[!h]
                \centering
                \includegraphics[scale=0.3]{logo.jpg}
            \end{figure}

            \vspace{0mm}
            \Large \textbf{{Meher Krishna Patel}}

            \small Created on : Octorber, 2017

            \vspace*{0mm}
            \small  Last updated : \MonthYearFormat\today


            %% \vfill adds at the bottom
            \vfill
            \small \textit{More documents are freely available at }{\href{http://pythondsp.readthedocs.io/en/latest/pythondsp/toc.html}{PythonDSP}}
        \end{titlepage}

        \clearpage
        \pagenumbering{roman}
        \tableofcontents
        \listoffigures
        \listoftables
        \clearpage
        \pagenumbering{arabic}

        
\pagestyle{plain}
 
\pagestyle{normal}
\phantomsection\label{\detokenize{index::doc}}



\chapter{Sistemas de radio}
\label{\detokenize{introduccion/sistemas:Sistemas-de-radio}}\label{\detokenize{introduccion/sistemas::doc}}
Los sistemas de comunicación a menudo implican transmitir un mensaje \(m(t)\) a través de un canal de ancho de banda finito, es decir, un canal donde solo se puede usar un rango limitado de frecuencias. Un buen ejemplo son las transmisiones de radio FM comerciales, generalmente restringidas a una banda de frecuencia entre \(85MHz\) y \(108MHz\) donde se transmiten múltiples estaciones, a cada una de las cuales se le asigna una banda \(\le 200 KHz\).

Dado que el mensaje que estamos interesados en transmitir a menudo tiene soporte en un rango diferente de frecuencias, como es el caso de las señales de audio sin procesar en el rango de audición humana (\(20 Hz - 20 kHz\)), la señal primero debe transladarse en frecuencia para satisfacer los requerimientos del canal particular de comunicación.

El siguiente diagrama en bloques presenta un sistema de radio basico.

\sphinxincludegraphics[width=731\sphinxpxdimen,height=276\sphinxpxdimen]{{radio1}.png}

El sistema de radio tiene como objetivo enviar y/o recibir información de dos lugares remotos.

En el caso representado, la información sera trasmitica por un enlace inalambrico realizado mediante dos antenas. El enlace atenua la señal enviada y, ademas, adiciona ruido a la señal.

Recordando el teorema de Shannon\sphinxhyphen{}Hartley, el cual establece cuál es la capacidad del canal, para un canal con ancho de banda finito y una señal continua que sufre un ruido gaussian:
\begin{equation*}
\begin{split}C=B\log _{2}\left(1+{\frac {S}{N}}\right)\end{split}
\end{equation*}
donde:

\(B\) es el ancho de banda del canal en Hertzios, \(C\) es la capacidad del canal (tasa de bits de información bit/s), \(S\) es la potencia de la señal útil, \(N\) es la potencia del ruido presente en el canal, que trata de enmascarar a la señal útil.

El teorema muestra el limite de la velocidad de transmision depende del ancho de banda y el ruido adicional al canal.


\section{Ejemplo de valores tipicos de BLU y STM}
\label{\detokenize{introduccion/sistemas:Ejemplo-de-valores-tipicos-de-BLU-y-STM}}
\sphinxincludegraphics[width=800\sphinxpxdimen,height=1200\sphinxpxdimen]{{AM_radio}.jpg}

\sphinxincludegraphics[width=1348\sphinxpxdimen,height=605\sphinxpxdimen]{{radio2}.png}

\sphinxincludegraphics[width=753\sphinxpxdimen,height=692\sphinxpxdimen]{{enlaceCband}.png}


\section{Enlace de radiocomunicaciones}
\label{\detokenize{introduccion/sistemas:Enlace-de-radiocomunicaciones}}
\sphinxincludegraphics[width=1120\sphinxpxdimen,height=554\sphinxpxdimen]{{radio3}.png}


\section{Cualidades de un receptor:}
\label{\detokenize{introduccion/sistemas:Cualidades-de-un-receptor:}}
Sensibilidad: capacidad de recibir señales débiles. Se mide como tensión en la entrada necesaria para obtener una relación determinada entre señal y ruido a la salida.

Selectividad: capacidad de rechazar frecuencias indeseadas. Se mide como cociente de potencias de entrada de las señales de frecuencias indeseadas y de la deseada que generan la misma señal de salida.

Fidelidad: capacidad de reproducir las señales de banda base para una distorsión especificada.

Margen dinámico: cociente entre niveles máximos y mínimos de potencia de entrada que garantizan funcionamiento correcto del receptor.

Liniealidad: la falta de linealidad produce intermodulación y modulación cruzada


\subsection{Sensibilidad:}
\label{\detokenize{introduccion/sistemas:Sensibilidad:}}
La sensibilidad del receptor determina el nivel de señal más débil que el receptor es capaz de recibir con una reproducción aceptable de la señal modulante original. La sensibilidad última del receptor se limita por el ruido generado dentro del propio receptor, siendo la relación señal a ruido y la potencia de la señal en la salida, indispensables en la determinación de la calidad de la señal demodulada. El ruido de salida es un factor importante en cualquier medición de sensibilidad.

La sensibilidad se define como el voltaje mínimo de entrada (portadora de RF), que producirá una relación de potencia señal a ruido (SNR) especificada generalmente a la salida de la sección demoduladora, generalmente se especifica en \(\mu V\). En algunos casos la portadora de RF se modula con un determinado índice y en otros se utiliza a la portadora de RF sin modular.

La potencia de ruido en un resistor esta dada por:
\begin{equation*}
\begin{split}N_{Res} = 4 k_B T R B\,[\frac{V^2}{Hz}]\end{split}
\end{equation*}
Donde \(k_B\) es la constante de Boltzmann (\(\approx \; 1,38064852 \times 10^{-23} J/K\)), \(T\) es la temperatura a la que se halla el resistor en Kelvin {[}\(K\){]}, y \(R\) su valor en Ohmios {[}\(\Omega\){]}.

Entonces la potencia disponible:
\begin{equation*}
\begin{split}N_{dis} = k_B T B\,[W]\end{split}
\end{equation*}

\begin{savenotes}\sphinxattablestart
\centering
\begin{tabulary}{\linewidth}[t]{|T|T|T|T|T|T|T|}
\hline
\sphinxstyletheadfamily &\sphinxstyletheadfamily 
BW
&\sphinxstyletheadfamily 
Pa ruido
&\sphinxstyletheadfamily 
V de ruido
&\sphinxstyletheadfamily 
Vseñal 20Db
&\sphinxstyletheadfamily 
Pseñal
&\sphinxstyletheadfamily 
Pseñal
\\
\hline
SERVICIO
&
\(KHz\)
&
\(pW\)
&
\(\mu V/50 \Omega\)
&
\(\mu V\)
&
\(pW\)
&
dBm
\\
\hline
TELEGRAFIA
&
0,2
&
8,28E\sphinxhyphen{}0
&
0,00643
&
0,0643
&
0,0000828
&
\sphinxhyphen{}131
\\
\hline
BLU
&
3
&
1,242E\sphinxhyphen{}05
&
0,02492
&
0,2492
&
0,001242
&
\sphinxhyphen{}119
\\
\hline
AM
&
10
&
0,0000414
&
0,04550
&
0,4550
&
0,00414
&
\sphinxhyphen{}114
\\
\hline
VHF
&
25
&
0,0001035
&
0,07194
&
0,7194
&
0,01035
&
\sphinxhyphen{}110
\\
\hline
RADIO E1
&
2000
&
0,00828
&
0,64343
&
6,4343
&
0,828
&
\sphinxhyphen{}91
\\
\hline
TV
&
6000
&
0,02484
&
1,11445
&
11,1445
&
2,484
&
\sphinxhyphen{}86
\\
\hline
RADIO STM1
&
30000
&
0,1242
&
2,49199
&
24,9199
&
12,42
&
\sphinxhyphen{}79
\\
\hline
\end{tabulary}
\par
\sphinxattableend\end{savenotes}

En receptores de AM se define la sensibilidad como el voltaje de la portadora mínimo de entrada, modulado en \(30\%\), con un tono de 1000 Hz, que produce una SNR especificada a la salida del detector de aproximadamente 10 dB, para el caso de receptores de televisión este valor es de aproximadamente 40dB. Para el caso de receptores de FM banda angosta se suelen definir básicamente 3 tipos de sensibilidades:


\subsubsection{Sensibilidad para 12 dB Sinad:}
\label{\detokenize{introduccion/sistemas:Sensibilidad-para-12-dB-Sinad:}}
A esta se la llama también Sensibilidad Útil y determina el nivel de señal de entrada de RF en el conector de antena que produce en la salida de audio una señal con una relación SINAD de \(12 dB\), donde será:
\begin{equation*}
\begin{split}SINAD = \frac{Señal + Ruido + Distorsión}{Ruido + Distorsión}\end{split}
\end{equation*}
en este caso se utiliza a la portadora de RF modulada al \(60 \%\) con un tono de \(1 KHz\).


\subsubsection{Sensibilidad para 20 dB de aquietamiento:}
\label{\detokenize{introduccion/sistemas:Sensibilidad-para-20-dB-de-aquietamiento:}}
Esta indica el nivel de señal de RF de entrada que produce un silenciamiento o atenuación del ruido de salida del receptor de 20 dB, en este caso la señal de entrada no se encuentra modulada. Los valores normales que se obtienen están en el orden de \(0,35 \mu V\) a \(0,5 \mu V\).


\subsubsection{Sensibilidad de apertura de silenciador:}
\label{\detokenize{introduccion/sistemas:Sensibilidad-de-apertura-de-silenciador:}}
El circuito silenciador (Squelch) en el receptor es el encargado de silenciar o enmudecer la salida de audio cuando no existe señal de entrada, este se debe habilitar cuando aparece una señal de entrada con un nivel mínimo (ajustable), este nivel mínimo con el cual se habilita la salida de audio es el que se conoce como Sensibilidad de Silenciador (Mute o Silenciador). El valor típico de sensibilidad se apertura está en el orden de \(0,18 \mu V\) a \(0,25 \mu V\), para receptores muy
sensibles.


\subsection{Selectividad:}
\label{\detokenize{introduccion/sistemas:Selectividad:}}
La selectividad es una medida de la capacidad del receptor para seleccionar la estación deseada y discriminar o atenuar señales de canales adyacentes no deseadas. La selectividad se determina por la respuesta en frecuencia que presentan algunos circuitos que anteceden al detector, especialmente los filtros de la sección de FI. El valor normalizado de rechazo de señales de canales adyacentes es tipicamente de \(60 dB\).

\sphinxincludegraphics[width=1300\sphinxpxdimen,height=875\sphinxpxdimen]{{selectividad}.png}

La determinación del rechazo de señales de canal adyacente en un receptor se puede realizar en forma estática o dinámica: En la forma dinámica se utilizan dos generadores de radiofrecuencia, uno se sintoniza a la frecuencia nominal del receptor con un nivel equivalente al de sensibilidad útil, el segundo generador se sintoniza a la frecuencia del canal adyacente cuyo rechazo se desea medir, modulado con un tono de 400 Hz y un índice del \(60\%\), se ajusta el nivel de salida de este
generador hasta que la relación SRD / RD se degrade de 12 a 6 dB, el rechazo se especifica por la diferencia en dB de los niveles de salida de los dos generadores.

\sphinxincludegraphics[width=1167\sphinxpxdimen,height=867\sphinxpxdimen]{{selectividad2}.png}


\subsubsection{Ancho de Banda:}
\label{\detokenize{introduccion/sistemas:Ancho-de-Banda:}}
El ancho de banda que debe presentar el receptor depende del tipo de servicio al que lo destinará, para el caso de AM con modulación de telefonía, el ancho de banda debe ser de 6 Khz, para AM comercial es de \(10 KHz\), para FM banda angosta debe ser de \(15 KHz\). En el receptor la etapa encargada de determinar el ancho de banda es la FI a través de los filtros que utiliza, como se ve mas adelante.


\subsection{Linealidad}
\label{\detokenize{introduccion/sistemas:Linealidad}}

\subsubsection{Distorsión Por Modulación Cruzada:}
\label{\detokenize{introduccion/sistemas:Distorsi_xf3n-Por-Modulaci_xf3n-Cruzada:}}
Si se inyectan simultáneamente señales deseadas y no deseadas, en transistores u otros dispositivos alinéales, estos producirán distorsión de tercer orden, la modulación de la amplitud sobre la señal no deseada se puede transferir a la portadora deseada. Esto se conoce como Modulación Cruzada.

La modulación cruzada crea problemas principalmente si la señal que se desea recibir es débil y se encuentra en un canal adyacente de una señal indeseada intensa, procedente de un transmisor cercano. Puede presentarse en la etapa mezcladora o en el amplificador de RF, por lo que el uso de FETs en lugar de BJTs es deseable en ambas etapas.

\sphinxincludegraphics[width=1791\sphinxpxdimen,height=834\sphinxpxdimen]{{linealidad1}.png}


\subsubsection{Intermodulación de Tercer Orden con dos tonos:}
\label{\detokenize{introduccion/sistemas:Intermodulaci_xf3n-de-Tercer-Orden-con-dos-tonos:}}
La distorsión por intermodulación también llamada distorsión de frecuencia, se produce en las primeras etapas del receptor, debido a la presencia de múltiples señales de RF de entrada y sus armónicos, mezcladas unas con otras y con la señal del oscilador local, produciendo en la salida frecuencias que no se encuentran presente en la entrada. Esto se produce por la alinealidad que presentan los elementos activos que se utilizan tanto en el amplificador de RF como en el mezclador. Cuando se
aplican en la entrada en forma simultánea dos señales o tonos de frecuencias F1 y F2 próximas y si sus amplitudes son tales que alcanzan la zona no lineal del amplificador de entrada, aparecen en la salida frecuencias resultado de la mezcla que no estaban presente en la entrada, tales como:
\begin{equation*}
\begin{split}Fs  =  f_{LO} ( nF1 mF2)\end{split}
\end{equation*}
La intermodulación de segundo orden genera componentes en la zona del segundo armónico y frecuencia diferencia (2f1, 2f2, f1 f2 , etc.), pudiendo presentar problemas en sistemas de banda ancha, en sistemas de banda angosta generalmente caen fuera de la banda. Los productos de Intermodulación de Tercer Orden frecuentemente caen dentro del ancho de banda, generando señales en la zona del tercer armonico y de las frecuencias de entrada, esto se puede ver en la siguiente figura

Las componentes de productos de intermodulación de tercer orden que caen fuera de la banda de interés son facilmente eliminados por los filtros que siguen al mezclador, pero los productos cruzados producidos cuando a la segunda armonica de una señal se le agrega la frecuencia fundamenta de otra señal .


\subsubsection{Productos de intermodulacion de tercer orden con dos tonos}
\label{\detokenize{introduccion/sistemas:Productos-de-intermodulacion-de-tercer-orden-con-dos-tonos}}
(2f1 \textendash{} f2) ó (2f2 \textendash{} f1) en la figura anterior, las componentes caen dentro de la banda original, siendo muy dificil su eliminación, donde el resultado será siempre 3, a esto se lo denomina Como la amplitud de los tonos de intermodulación es proporcional al cubo de la amplitud de la señal de entrada, la potencia de estos tonos sera también proporcional al cubo de la potencia de la señal de entrada, por lo que la potencia de salida de los productos de intermodulación resulta ser proporcional al
cubo de la potencia de salida de la señal (I3 Po3). La respuesta característica típica de distorsión de tercer orden en función de la potencia de entrada se puede en la figura siguiente:

\sphinxincludegraphics[width=1067\sphinxpxdimen,height=567\sphinxpxdimen]{{linealidad2}.png}

Como se puede ver la pendiente de I3 es tres veces mayor que la pendiente de Po, incrementandose 3 dB por cada dB de incremento en la potencia de entrada, se denomina Punto de Intercepción de Tercer Orden (PI3) al punto ficticio donde se cruzan las rectas de respuesta lineal y de tercer orden, en este punto se igualan la potencia de salida lineal con la potencia de salida de intermodulación de tercer orden, generalmente este punto se encuentra entre 10 y 16 dB por encima del punto de compresión
de \(1 dB\). Los valores de F1 y F2 más críticos son aquellos que están próximos a la frecuencia de recepción. Cuando se mide a un receptor el rechazo de intermodulación de 3o orden, durante el proceso de homologación, las frecuencias F1 y F2 que se utilizan están separadas 1 y 2 canales de la frecuencia deseada.


\chapter{Receptores de radiofrecuencia}
\label{\detokenize{introduccion/sistemas:Receptores-de-radiofrecuencia}}
Un receptor es un dispositivo capaz de aceptar y demodular una señal de radio frecuencia, a fin de obtener la información transportada en ella. La señal de entrada al receptor generalmente presenta una energia extremadamente baja, por lo tanto, un receptor típico debe ser capaz de amplificar la señal de entrada por un factor del orden de algunos cientos, para que esta tenga suficiente amplitud para ser útil.


\section{Historia de los receptores de radio}
\label{\detokenize{introduccion/sistemas:Historia-de-los-receptores-de-radio}}
Cuando el superheterodino estaba en período de perfeccionamiento, se comercializó un equipo de radio que, si bién no tenía la capacidad del heterodino en cuanto a sensibilidad ni a selectividad, en aquella época era lo más avanzado del momento. Hablamos del receptor de Radiofrecuencia Sintonizada.

Este receptor fue muy popular entre los años veinte y los años treinta. Aunque se comenzó a fabricar con triodos, con el desarrollo de la válvula tetrodo y la aparición en escena de los nuevos pentodos se facilitaron mucho las cosas para que el receptor de radiofrecuencia sintonizada se presentase al público en general, y con mucho éxito en el mercado.

Para entender como evolucionó la tecnología del momento debemos empezar desde el principio. Dejando de lado los receptores más básicos, algunos de ellos ya estudiados en otros artículos y que en la práctica no tuvieron la acogida del gran público, nos centraremos en el que puede considerarse como el primer receptor de gran éxito comercial de la historia en sus diferentes versiones. Es el llamado receptor de Radiofrecuencia Sintonizada, denominado también por algunos fabricantes (entre ellos
Philips) receptor a Superinductancia.


\subsection{Amplificadores sintonizados}
\label{\detokenize{introduccion/sistemas:Amplificadores-sintonizados}}
Con la invención de las válvulas de vacío, en lo primero que se pensó fue en amplificar la señal de RF mediante uno o dos triodos para conseguir la ansiada sensibilidad. Para que el receptor, además, disfrutara de una buena selectividad, los pasos amplificadores tendrían que ser “selectivos”, es decir, que solo amplificaran una determinada frecuencia; aquella que se quería oir.

La escucha de emisoras se realiza mediante un simple auricular, la figura siguiente ilustra el circuito.

\sphinxincludegraphics[width=588\sphinxpxdimen,height=210\sphinxpxdimen]{{receprfs1low}.png}

Con este receptor la señal de RF es amplificada mediante los triodos V1 y V2 montados en cascada, antes de ser demodulada. Después de la demodulacion efectuada por V3, la señal de BF resultante es amplificada por el propio triodo V3 antes de aplicarse al auricular. Con este tipo de receptor se conseguiría mejorar la sensibilidad y una selectividad, pero había ciertos problemas.

Lógicamente, con tres capaciotores variables la sintonía sería muy laboriosa para determinadas emisoras, sobre todo las más débiles. Se tendría que ir ajustando capacitores por capacitores hasta conseguir que los tres circuitos resonantes sintonizaran idéntica frecuencia, y que esta coincidiera con la de la emisora que se quería oir. La operacion era complicada para un usuario normal. Además, debido a la “capacidad parásita” placa\sphinxhyphen{}rejilla y rejilla\sphinxhyphen{}cátodo de los triodos, el circuito a menudo
adolecía de inestabilidad. Estas capacidades parásitas son algo inherente a todas las válvulas triodo.

En el triodo podemos aplicar un razonamiento similar. Sabemos que la placa y la rejilla son dos elementos metálicos que pueden hacer las veces de armaduras de un condensador. Estos elementos están separados por un aislante, el vacío, que actúa como un dieléctrico, por lo que el efecto es el mismo que el que produciría un condensador conectado entre placa y rejilla (Cgp). Lo mismo podemos decir de la rejilla y el cátodo (Cgk), e incluso de la placa y el cátodo (Cpk) también, aunque esta última
afecta en menor grado que las anteriores al funcionamiento del triodo ya que justo en medio se encuentra la rejilla, la cual establece cierta separación.

Estas capacidades parásitas producían una realimentación o reacción en el circuito, lo que provocaba que a partir de determinadas frecuencias el triodo se volviera completamente inestable y la recepción de emisoras se convirtiera en una “jaula de grillos” por la cantidad de silbidos y ruidos que se producían.

El primer problema, el de los tres circuitos resonantes independientes que causaban tantas molestias para sintonizar una determinada emisora, se mitigó con algo muy sencillo: el capacitor variable en “tandem”. Se trata simplemente de “sincronizar” el desplazamiento físico de los tres condensadores variables, de manera que la frecuencia de resonancia de cada circuito sea siempre la misma para los tres, fuera la que fuera la posición de las armaduras móviles de los condensadores.

Para que se pudiera distinguir en los esquemas electrónicos un capacitor en tandem de los que se montaban de manera independiente, los primeros se representaban unidos mediante una linea discontinua, tal y como se indica en el dibujo que sigue.

\sphinxincludegraphics[width=588\sphinxpxdimen,height=213\sphinxpxdimen]{{receprfs2low}.png}

Para mejorar el problema de la inestabilidad, silbidos y ruidos causados por las capacidades parásitas del triodo, la solución fue el neutrodino.


\subsection{El neutrodino}
\label{\detokenize{introduccion/sistemas:El-neutrodino}}
El siguiente paso fué anular los efectos de las capacidades parásitas del triodo. En realidad, y debido a que el circuito de placa de las válvulas no estaba constituido por resistencias puras sino por bobinas y condensadores, el problema que introducían las capacidades parásitas del triodo, que en un principio y por la configuración del circuito debería tratarse de una realimentación negativa, tenía cierto componente de realimentación positiva (como ocurría en el receptor a reacción). Esto era
suficiente para estropear el invento y evitar una recepción limpia y nítida de las señales.

Se usaron capacidades estratégicamente colocadas para anular las capacidades parásitas, o mejor dicho, para anular el efecto que causan estas últimas. Estos capacitores introducían en el circuito una nueva realimentación, pero en este caso dicha realimentación se oponía a la que introducían las capacidades parásitas.

\sphinxincludegraphics[width=588\sphinxpxdimen,height=226\sphinxpxdimen]{{receprfs3low}.png}

La realimentación o contra\sphinxhyphen{}realimentación introducida por los nuevos condensadores (C4 y C5 en el esquema superior) estaba justo en “oposición de fase” con la provocada por las capacidades parásitas. Además, estos condensadores solían ser de capacidad variable, aunque bastante más pequeños que los usados para la sintonía, lo que permitía un ajuste exacto del nivel de contra\sphinxhyphen{}realimentación.

Se había conseguido “neutralizar” el efecto de las capacidades parásitas del triodo. Por esta razón, a este diseño particular de receptor, inventado por el ingeniero y físico norteamericano Louis Alan Hazeltine, se le llamó “receptor neutrodino”.

El neutrodino era un receptor muy estable, libre de ruidos y silbidos y de cómoda sintonía para el usuario. Sin embargo, la llegada de la válvula pentodo lo remplazo.


\subsection{Receptores de radiofrecuencias sintonizados con pentodos}
\label{\detokenize{introduccion/sistemas:Receptores-de-radiofrecuencias-sintonizados-con-pentodos}}
El neutrodino mejoró sensiblemente el funcionamiento y las características generales del receptor de radiofrecuencia sintonizada con triodos. No obstante, la sustitución de estos últimos por los recien inventados pentodos puso el listón aún más alto e hizo innecesario usar la neutrodinación.

Efectivamente, con la introducción de dos rejillas más entre la de control y la placa del triodo las capacidades parásitas se redujeron sotensiblemente, de manera que dejaron de causar los problemas que tantos quebraderos de cabeza les dió a los diseñadores. Para conseguir un receptor estable ya no hacía falta usar capacitores neutralizadores.

\sphinxincludegraphics[width=588\sphinxpxdimen,height=226\sphinxpxdimen]{{receprfs4low}.png}

Pero por desgracia, esto no solucionó definitivamente los inconvenientes del receptor de radiofrecuencia sintonizada. Mantener exactamente la misma frecuencia de sintonía en los tres circuitos resonantes independientemente de la posición de las placas del tandem no era sencillo ya que el proceso de fabricación introducía pequeñas diferencias en los componentes que hacían que no fueran completamente idénticos. Además, el paso del tiempo y el envejecimiento de los materiales utilizados producía
irremisiblemente desajustes que llevaban al receptor a una pérdida de sensibilidad.

Por estas y otras razones, cuando apareció en escena el receptor superheterodino todos los demás se dejaron de fabricar de manera casi instantánea. Las características del nuevo modelo superaron con mucho a todos los demás, lo que supuso que a partir de entonces todos los receptores pasaran a fabricarse con la técnica del llamado “batido de frecuencias” usada en el superheterodino. Pero eso será un asunto que trataremos en otro artículo posterior.


\chapter{Receptor Homodino}
\label{\detokenize{introduccion/sistemas:Receptor-Homodino}}
Receptor homodino, es un tipo de receptor donde el demodulador opera a la frecuencia de RF. “Homodinas” significa una única frecuencia, en contraste con la doble frecuencias empleadas en la detección heterodina. La siguiente figura ilustra un receptor super\sphinxhyphen{}homodino, ya que se le llama super al agregar amplificadores en la etapa de RF.

\sphinxincludegraphics[width=919\sphinxpxdimen,height=270\sphinxpxdimen]{{homodino}.png}

En general, la sintonia se logra con la resonancia de un circuito LC. El ancho de banda de un circuito LC cargado depende de lala resistencia del circuito, dado por la fuente, la carga y las perdidas, y del valor de la admitancia en resonancia. Si varia la admitancia de resonancia para lograr la sintonia, y suponiendo que la resistencia del circuito no se modifica, tambien se modifica el Q del sintonizado (junto con el ancho de banda del circuito). Por lo tanto, la selectividad obtenida varía en
función de la frecuencia de recepción.

\sphinxincludegraphics[width=919\sphinxpxdimen,height=270\sphinxpxdimen]{{homodino2}.png}

Al operar todas las etapas de RF a la misma frecuencia. existe la posibilidad de oscilaciones por acoplamientos parásitos entre entrada y salida,

No es aconsejable si el margen de frecuencias a recibir es ancho, ya que hacen falta varios filtros de banda agudos y variables.


\section{Ejemplo: Receptor de banda lateral unica (SSB, Single Side Band)}
\label{\detokenize{introduccion/sistemas:Ejemplo:-Receptor-de-banda-lateral-unica-(SSB,-Single-Side-Band)}}
En el ejemplo se muestra un detector coherente para SSB.

\sphinxincludegraphics[width=613\sphinxpxdimen,height=647\sphinxpxdimen]{{DetCoherente}.png}

La solución para evitar que las frecuencias imagenes no sean detectadas es el uso de un detector coherente con mezclador I/Q


\section{Receptor Regenerativo 433MHz}
\label{\detokenize{introduccion/sistemas:Receptor-Regenerativo-433MHz}}
\sphinxincludegraphics[width=1024\sphinxpxdimen,height=682\sphinxpxdimen]{{FS1000A-and-XY-MK-5V-1024x682}.jpg}

\sphinxincludegraphics[width=392\sphinxpxdimen,height=348\sphinxpxdimen]{{FS1000_circuit}.jpg}

\sphinxincludegraphics[width=837\sphinxpxdimen,height=313\sphinxpxdimen]{{xy-mk-5V_circuit}.jpg}


\chapter{Receptor Heterodino}
\label{\detokenize{introduccion/sistemas:Receptor-Heterodino}}
Heterodinar significa mezclar dos frecuencia en una etapa alineal (mezclador) a fin de obtener la suma o diferencia de las dos frecuencias de entrada. Los receptores superheterodinos basan su funcionamiento en la utilización de una o mas etapas mezcladoras, estas trasladan la frecuencia de recepción a un valor de frecuencia normalizado, generalmente menor, denominado Frecuencia Intermedia (FI), para poder mezclar o heterodinar dos señales se debe disponer de una etapa mezcladora y un oscilador
local, este último es además el encargado de seleccionar la frecuencia que se desea recibir.

La idea es convertir todas las frecuencias a recibir a una unica frecuencia llamada “Frecuencia Intermedia”. Esto permite que el receptor pueda recibir distintas frecuencias con un solo demodulador, donde el mayor esfuerzo en filtrado y amplificación en alta frecuencia se hace a la frecuencia intermedia. La sintonía se lleva a cabo modificando la frecuencia del oscilador (oscilador local) y la del filtro de entrada (si el margen de frecuencias a recibir es amplio).


\section{Super\sphinxhyphen{}heterodino}
\label{\detokenize{introduccion/sistemas:Super-heterodino}}
Un receptor super\sphinxhyphen{}heterodino se diferencia de un heterodino común gracias a una serie de mejoras como un amplificador de RF de entrada, un circuito de AGC y otras etapas que optimizan el funcionamiento.

A estos receptores básicamente se los puede clasificar en Receptores de Simple Conversión y Receptores de Doble Conversión.


\section{Super\sphinxhyphen{}heterodino de simple conversión.}
\label{\detokenize{introduccion/sistemas:Super-heterodino-de-simple-conversi_xf3n.}}
El diagrama en bloques de un receptor de simple conversión se ve a continuación:

\sphinxincludegraphics[width=917\sphinxpxdimen,height=380\sphinxpxdimen]{{SuperHeterodino1}.png}


\subsection{Ejemplo practico : Receptor de radiodifusión AM.}
\label{\detokenize{introduccion/sistemas:Ejemplo-practico-:-Receptor-de-radiodifusi_xf3n-AM.}}
\begin{DUlineblock}{0em}
\item[] Las frecuencias de la banda de AM comerncial:
\end{DUlineblock}
\begin{quote}
\begin{equation*}
\begin{split}F_{RF_{min}} = 520 KHz\end{split}
\end{equation*}\end{quote}

\begin{DUlineblock}{0em}
\item[] 
\end{DUlineblock}
\begin{quote}
\begin{equation*}
\begin{split}F_{RF_{max}} = 1630 KHz\end{split}
\end{equation*}\end{quote}

Para la sintonia de la los distintos canales se emplea un receptor superheteroino de simple conversión, donde la frecuencia \(F_{IF} = 455 KHz\) y el ancho de banda del canal es de \(\Delta F_{IF} = 10 KHz\), esto ultimo usando un filtro cerámico (SFU455A).

El oscilador local puede tomar cualquiera de las siguientes frecuencias:
\begin{equation*}
\begin{split}f_{osc} = f_{RF} + f_{IF}\end{split}
\end{equation*}\begin{equation*}
\begin{split}f_{osc} = f_{RF} - f_{IF}\end{split}
\end{equation*}
Para este ejemplo, la emplearemos la primer opción: \(f_{osc_{min}} = 975 kHz\) y \(f_{osc_{min}} = 2085 kHz\).

El siguiente diagrama muestra como se realiza la recepcion del canal de \(f_{RF} = 1MHz\).

\sphinxincludegraphics[width=932\sphinxpxdimen,height=509\sphinxpxdimen]{{frecuenciasAM}.png}

La señal de RF que que proviene de la antena ingresa a la etapa de RF. En la etapa de RF, el filtro de RF se encuentra sintonzado a la frecuencia del canal que se desea sintonizar (\(f_{RF} = 1 MHz\)). Este filtro al estar sintonizado a la frecuencia del canal no atenua el canal deseado. Tambien, la señal es amplificada por el amplificador de RF en esta etapa.

La señal luego es mezclada con la señal del oscilador local, que debe estar operando a \(f_{osc} = 1,455 MHz\).

El mezclador ideal tiene una respuesta artimetica, de tal forma, la señal de salida se obtiene de la siguiente ecuación:
\begin{equation*}
\begin{split}v_{IF}(t) = K \cdot v_{OL}(t) \cdot v_{RF}(t)\end{split}
\end{equation*}
Si las señales de entrada son, en el caso más simple, ondas de tensión senoidales, entonces:
\begin{equation*}
\begin{split}v_{IF}(t) = K \cdot cos(\omega_{OL} \cdot t) \cdot cos(\omega_{RF} \cdot t)\end{split}
\end{equation*}
Para resolver esta ecuacion, podemos emplear la identidad trigonometrica:
\begin{equation*}
\begin{split}cos(\alpha) \cdot cos(\beta) = \frac{1}{2} \cdot [cos(\alpha + \beta) +  \cdot cos(\alpha - \beta)  ]\end{split}
\end{equation*}
Empleando la identidad:
\begin{equation*}
\begin{split}v_{IF}(t) = K \cdot \frac{1}{2} \cdot [cos( (\omega_{OL}+ \omega_{RF})  \cdot t) + cos( (\omega_{OL}-\omega_{RF} ) \cdot t) ]\end{split}
\end{equation*}
En función de la frecuencia, entonces:
\begin{equation*}
\begin{split}v_{IF}(t) = K \cdot \frac{1}{2} \cdot [cos( 2 \pi \cdot |f_{OL}+ f_{RF}|  \cdot t) + cos( 2 \pi \cdot |f_{OL}-f_{RF} | \cdot t) ]\end{split}
\end{equation*}
Como vemos, la señal de salida esta compuesta por dos componentes:
\begin{equation*}
\begin{split}|f_{OL}+ f_{RF}| = 1455 KHz + 1000 KHz = 2455 KHz\end{split}
\end{equation*}\begin{equation*}
\begin{split}|f_{OL}- f_{RF}| = 1455 KHz - 1000 KHz = 455 KHz\end{split}
\end{equation*}
La salida de un mezclador contiene la suma y la diferencia de las dos frecuencias de entrada, \(f_{OL} \pm f_{RF}\).

La componente de mas alta frecuencia (\(f_{OL} + f_{RF}\)) es atenuada por el filtro de IF. El ancho de banda de la señal resultante esta realcionada con la respuesta del filtro de RF, que por cuestiones constructivas, no tiene la suficiente selectividad como para eliminar los canales adyacentes. Esta selectividad se obtine del filtro de IF.

\begin{DUlineblock}{0em}
\item[] Notar que calculamos los modulos de las componentes de las frecuencias. En este caso la respuesta es para la componente positiva. Pero dada la respuesta del mezclador, la frecuencia \(f_{RF} = 1910 KHz\) también tiene respuesta en la frecuencia \(455KHz\), la cual corresponde a la componente negativa del modulo.
\item[] Esta señal es no deseada y se llama “frecuencia imagen” o \(f_{imagen}\), debido a la simetría entre ambas frecuencias detectables respecto a \(f_{OL}\). Esta señal a la salida del mezclador no puede ser elmimada ya que su respuesta se superpone a la señal deseada, por lo tanto, debe ser eliminada antes de ingresar al mezclador.
\end{DUlineblock}

La sensibilidad a la frecuencia imagen puede ser minimizada o bien mediante un filtro sintonizable que preceda al mezclador, o bien mediante un circuito mezclador mucho más complejo.

La señal de frecuencia imagen está separada de la señal que se desea recibir en un valor igual a dos veces la FI, si esta señal de frecuencia imagen llega al mezclador, el receptor ya no será capaz de eliminarla. El rechazo de señales de frecuencia imagen generalmente se busca que sea menor a 60 dB (dependiendo esto del tipo de servicio), pudiendo ser necesario un valor mayor.

En la siguiente figura se ilustra el rechazo de frecuencia imagen de una etapa de RF.

\sphinxincludegraphics[width=751\sphinxpxdimen,height=526\sphinxpxdimen]{{rechazoImagen}.png}


\subsection{Sensibilidad del receptor}
\label{\detokenize{introduccion/sistemas:Sensibilidad-del-receptor}}
Para que el receptor sea capaz de recibir señales de pequeña amplitud, el aporte de ruido de este debe ser también pequeño, fundamentalmente debe tener una cifra de ruido baja, idealmente 1, para esto se suele emplear amplificadores de bajo ruido (LNA).

La fórmula de Friis se utiliza para calcular el factor de ruido total de etapas en serie, cada una con su respectivas pérdidas o ganancias y su respectiva factor de ruido. El factor de ruido total puede ser utilizado posteriormente para calcular la cifra de ruido total. El factor de ruido total se calcula mediante la siguiente fórmula:
\begin{equation*}
\begin{split}F_{total} = F_1 + \frac{F_2-1}{G_1} + \frac{F_3-1}{G_1 G_2} + \frac{F_4-1}{G_1 G_2 G_3} + ... + \frac{F_n - 1}{G_1 G_2 ... G_{n-1}}\end{split}
\end{equation*}
donde \(F_n\) y \(G_n\) son el factor de ruido y la ganancia en potencia disponible, respectivamente, de la enésima etapa.
\begin{equation*}
\begin{split}F_{receptor} = F_{LNA} + \frac{(F_{resto}-1)}{G_{LNA}}\end{split}
\end{equation*}
donde \(F_{resto}\) es el factor de ruido total de las etapas subsecuentes. De acuerdo a la ecuación, la cifra de ruido total, \(F_{receptor}\), es dominada por la cifra de ruido del amplificador de bajo ruido, \(F_{LNA}\), si la ganancia es lo suficientemente alta.


\subsection{Ejemplo de receptor de conversión simple}
\label{\detokenize{introduccion/sistemas:Ejemplo-de-receptor-de-conversi_xf3n-simple}}
Receptor de radiodifusión en FM (VHF, modulación en FM de banda ancha) con sintonía sintetizada con PLL:
\begin{equation*}
\begin{split}f_{RF_{min}} = 87,5 MHz\end{split}
\end{equation*}\begin{equation*}
\begin{split}f_{RF_{max}} = 108 MHz\end{split}
\end{equation*}\begin{align*}\!\begin{aligned}
f_{IF_{1}} = 10,7 MHz\\
El filtros realizado con filtro ceramico.\\
\end{aligned}\end{align*}\begin{equation*}
\begin{split}\Delta f_{IF_{2}} = 250 kHz\end{split}
\end{equation*}\begin{equation*}
\begin{split}f_{OL_{1min}} = 98,2 MHz\end{split}
\end{equation*}\begin{equation*}
\begin{split}f_{OL_{1max}} = 118,7 MHz\end{split}
\end{equation*}
\sphinxincludegraphics[width=1281\sphinxpxdimen,height=315\sphinxpxdimen]{{dobleConver2}.png}

Ejemplo de circuito integrado super\sphinxhyphen{}heterodino de simple conversión: MAX1471.


\section{Super\sphinxhyphen{}heterodino de doble conversión.}
\label{\detokenize{introduccion/sistemas:Super-heterodino-de-doble-conversi_xf3n.}}
En el receptor de simple conversión, la selectividad del receptor está fijada por la del filtro de IF. Si aumenta \(f_{IF}\) aumenta su ancho de banda (para igual Q) y, por tanto, disminuye la selectividad del receptor. Para solucionar este problema hay dos soluciones posibles:

Usar filtros de más calidad (filtros cerámicos de alta calidad o filtros de cristal de cuarzo en vez de cerámicos).

Usar una estructura de conversión múltiple (doble o triple) como el superheterodino de doble conversión.

\sphinxincludegraphics[width=1281\sphinxpxdimen,height=315\sphinxpxdimen]{{dobleConver1}.png}

Dos frecuencias intermedias:

La primera frecuencia intermedia, \(f_{IF_1}\), se elige relativamente alta para conseguir buen rechazo a la frecuencia imagen. La segunda frecuencia intermedia, \(f_{IF_2}\), se elige relativamente baja para obtener una buena selectividad.


\subsection{Primer oscilador variable y primera IF constante}
\label{\detokenize{introduccion/sistemas:Primer-oscilador-variable-y-primera-IF-constante}}
\sphinxincludegraphics[width=1281\sphinxpxdimen,height=315\sphinxpxdimen]{{dobleConver2}.png}

Mejor solución si el margen de variación de \(f_{RF}\) es grande. El oscilador de más alta frecuencia es el variable, esto podria tener posibles problemas de estabilidad térmica. La solución es usar PLLs o DDSs.


\subsection{Primer oscilador constante y primera IF variable:}
\label{\detokenize{introduccion/sistemas:Primer-oscilador-constante-y-primera-IF-variable:}}
\sphinxincludegraphics[width=1281\sphinxpxdimen,height=315\sphinxpxdimen]{{dobleConver3}.png}

El oscilador de más alta frecuencia es de frecuencia fija (mejor desde el punto de vista de la estabilidad térmica). Solución sólo adecuada si el margen de variación de \(f_{RF}\) es pequeño. En caso contrario, existen problemas con el ruido y con el margen dinámico, ya que toda la banda a recibir es procesada por los amplificadores de RF y 1 IF, que deben ser de banda ancha.


\subsection{Receptor de radioaficionado de la banda de 2 m (VHF, modulación en FM de banda estrecha):}
\label{\detokenize{introduccion/sistemas:Receptor-de-radioaficionado-de-la-banda-de-2-m-(VHF,-modulaci_xf3n-en-FM-de-banda-estrecha):}}\begin{equation*}
\begin{split}f_{RF_{min}} = 144 MHz\end{split}
\end{equation*}\begin{equation*}
\begin{split}f_{RF_{max}} = 146 MHz\end{split}
\end{equation*}\begin{equation*}
\begin{split}f_{IF_{1}} = 10,7 MHz\end{split}
\end{equation*}\begin{align*}\!\begin{aligned}
f_{IF_{2}} = 455  kHz\\
Ambos filtros para las etapas intermedias realizados con filtros ceramicos.\\
\end{aligned}\end{align*}\begin{equation*}
\begin{split}\Delta f_{IF_{2}} = 15  kHz\end{split}
\end{equation*}\begin{equation*}
\begin{split}f_{OL_{1min}} = 154,7 MHz\end{split}
\end{equation*}\begin{align*}\!\begin{aligned}
f_{OL_{1max}} = 156,7 MHz\\
Se sintoniza empleando un PLL (phase locked loop).\\
\end{aligned}\end{align*}\begin{equation*}
\begin{split}f_{OL_{2}} = 10,245 MHz\end{split}
\end{equation*}
\sphinxincludegraphics[width=1306\sphinxpxdimen,height=584\sphinxpxdimen]{{dobleConversion}.png}


\chapter{Ganancia de potencia y definiciones}
\label{\detokenize{introduccion/sistemas:Ganancia-de-potencia-y-definiciones}}

\section{Potencia en dBm}
\label{\detokenize{introduccion/sistemas:Potencia-en-dBm}}
La potencia en los sistemas de comunicación a menudo se mide en la escala ‘dBm’, o la potencia de referencia medida en relación con a \(1 mW\).

P.ej. un nivel de potencia de 10 mW puede expresarse como 10 dBm.


\section{Potencia en un cuadripolo}
\label{\detokenize{introduccion/sistemas:Potencia-en-un-cuadripolo}}
\sphinxincludegraphics[width=503\sphinxpxdimen,height=316\sphinxpxdimen]{{power}.png}

Definimos:

\(P_{in}\) : Potenicia de entrada

\(P_{L}\) : Potenicia en la carga

\(P_{av,s}\) : Potenicia máxima disponible de la fuetne (available power)

En una fuente de tensión para valores pico:
\begin{equation*}
\begin{split}P_{av,s} = \frac{v_s^2}{8 r_g}\end{split}
\end{equation*}
En una fuente de corriente para valores pico:
\begin{equation*}
\begin{split}P_{av,s} = \frac{i_s^2 r_g}{8}\end{split}
\end{equation*}

\subsection{Ganancia de Potencia}
\label{\detokenize{introduccion/sistemas:Ganancia-de-Potencia}}\begin{equation*}
\begin{split}G_P = \frac{P_L}{P_{in} }\end{split}
\end{equation*}

\subsection{Ganancia de trasducción}
\label{\detokenize{introduccion/sistemas:Ganancia-de-trasducci_xf3n}}\begin{equation*}
\begin{split}G_T = \frac{P_L}{P_{av,s} }\end{split}
\end{equation*}\begin{equation*}
\begin{split}G_P = \frac{P_L}{P_{in} }\end{split}
\end{equation*}
\sphinxurl{http://rfic.eecs.berkeley.edu/~niknejad/ee142\_fa05lects/pdf/lect4.pdf}


\chapter{Técnicas de adaptación}
\label{\detokenize{adaptacion/adaptacion:T_xe9cnicas-de-adaptaci_xf3n}}\label{\detokenize{adaptacion/adaptacion::doc}}
\sphinxstylestrong{Adaptador de impedancia.}

Un adaptador de impedancia, en este caso un el cuadripolo colocado en cascada en el circuito, modifica la resistencia de carga dada \(R_L\) a una un valor dado de entrada o la resistencia de entrada a un valor dado de salida.

\sphinxincludegraphics[width=1168\sphinxpxdimen,height=315\sphinxpxdimen]{{cuadripolo}.png}

Dependiendo del uso, estos valores de entrada o salida se ajustan para lograr distintos objetivos. A continuación se listan los mas frecuentes.

\sphinxstylestrong{Transferencia de energia óptima}: maximiza la transferencia de energia desde la fuente (por ejemplo, una antena) y la carga (por ejemplo, un amplificador).

\sphinxstylestrong{Cifra de ruido óptima}: amplificadores que agreguen la menor cantidad de ruido a una señal mientras realizan la amplificación. Esta depende de la impedancia presentada al dispositivo activo.

\sphinxstylestrong{Criterio de estabilidad}: donde se busca la estabilidad del sistema.

\sphinxstylestrong{Reflexiones mínimas en las líneas de transmisión}: Las reflexiones causan dispersión e interferencia y dan como resultado una impedancia de entrada sensible cuando se mira en la línea de transmisión (cambia con la distancia).

\sphinxstylestrong{Eficiencia óptima}: los amplificadores de potencia obtienen la máxima eficiencia cuando utilizamos la mayor oscilación de voltaje posible en el nodo de salida de los elentos activos (drain o colector), lo que requiere que hagamos coincidir la carga con un valor que satisfaga las condiciones de potencia de carga y oscilación de carga.

\sphinxstylestrong{Cuadripolo de parametros admitancia}

El circuito de un cuadripolo admitancia se muestra en la figura.

\sphinxincludegraphics[width=668\sphinxpxdimen,height=198\sphinxpxdimen]{{admitancia}.png}

Las ecuaciones del cuadripolo en función de los parametros de admitancia y tensiones del circuito:
\begin{equation*}
\begin{split}i_i = v_i\cdot y_{11} + v_o \cdot y_{12}\end{split}
\end{equation*}\begin{equation*}
\begin{split}i_o = v_i\cdot y_{21} + v_o \cdot y_{22}\end{split}
\end{equation*}
\sphinxincludegraphics[width=803\sphinxpxdimen,height=209\sphinxpxdimen]{{admiEntrada}.png}

Del circuito, se puede calcular la admitacia de entrada dada una admitancia de salida \(y_{L}\):
\begin{equation*}
\begin{split}y_{in}=y_{11}- \frac {y_{12}y_{21}}{y_{22}+y_{L}}\end{split}
\end{equation*}
\sphinxincludegraphics[width=750\sphinxpxdimen,height=212\sphinxpxdimen]{{admiSalida}.png}

Del circuito, se puede calcular la admitacia de entrada dada una admitancia de salida \(y_{L}\):
\begin{equation*}
\begin{split}y_{out}=y_{22}- \frac {y_{12}y_{21}}{y_{11}+y_{g}}\end{split}
\end{equation*}

\section{Circuitos resonantes.}
\label{\detokenize{adaptacion/adaptacion:Circuitos-resonantes.}}
Ademas de realizar adaptaciones de impedancia, los sistemas de RF precisan filtros pasabanda para atenuar las bandas de frecuencias no deseada, como la de frecuencia imagen. Por su flexibilidad, los circuitos resonantes permiten diseñar filtros pasabanda fijos o variables.

En su forma mas básica, estan formados por elementos reactivos (inductancias y capacitancia). Estos circuitos pueden ser relizados por elementos de constantes concetradas como inductores o capacitores, elementos de constantes distribuidas, como los obtenidos de las líneas transmisión o elementos resonantes como cristales piezoeléctricos.

A continuación, se analizaran circuitos resonantes simples formados por inductancias y capacitancia en paralelo y en serie.


\subsection{Factor de selectidad \protect\(Q\protect\)}
\label{\detokenize{adaptacion/adaptacion:Factor-de-selectidad-Q}}
El factor de selectividad es un parámetro que mide la relación entre la energía reactiva que almacena y la energía que disipa durante un ciclo completo de la señal. Este parametro esta relacionado con el ancho de banda. Un alto factor \(Q\) indica una tasa baja de pérdida de energía en relación a la energía almacenada por el resonador. Es un parámetro importante para los osciladores, filtros y otros circuitos sintonizados, pues proporciona una medida de lo selectiva que es su resonancia.

El factor de selectividad entonces se calcula como:
\begin{equation*}
\begin{split}Q_o = \frac{Pot_{reactiva}}{ Pot_{activa}}\end{split}
\end{equation*}

\subsubsection{Factor de selectidad en circuito paralelo}
\label{\detokenize{adaptacion/adaptacion:Factor-de-selectidad-en-circuito-paralelo}}
En un circuito paralelo conformado por una reactancia y una resistencia, la tensión es un parametro comun para ambos componentes, por lo tanto, las potencias las debemos calcular en función de este.

En un circuito RL:
\begin{equation*}
\begin{split}Q_o =  \frac{ \frac{v_g^2}{w_o \cdot L} }{ \frac{v_g^2}{R} } = \frac{R}{w_o \cdot L}\end{split}
\end{equation*}
En un circuito RC:
\begin{equation*}
\begin{split}Q_o = \frac{v_g^2 \cdot w_o \cdot C }{ \frac{v_g^2}{R} } = R \cdot w_o \cdot C\end{split}
\end{equation*}

\subsubsection{Factor de selectidad en circuito serie}
\label{\detokenize{adaptacion/adaptacion:Factor-de-selectidad-en-circuito-serie}}
En un circuito serie conformado por una reactancia y una resistencia, la corriente es un parametro comun para ambos componentes, por lo tanto, las potencias las debemos calcular en función de este.

En un circuito RC:
\begin{equation*}
\begin{split}Q_o =  \frac{ \frac{i_g^2}{w_o \cdot C} }{i_g^2 \cdot R } = \frac{1}{R \cdot w_o \cdot C}\end{split}
\end{equation*}
En un circuito RL:
\begin{equation*}
\begin{split}Q_o =  \frac{i_g^2 \cdot w_o \cdot L }{ i_g^2 \cdot R } = \frac{w_o \cdot L}{R }\end{split}
\end{equation*}

\subsection{Circuito resonante RLC paralelo.}
\label{\detokenize{adaptacion/adaptacion:Circuito-resonante-RLC-paralelo.}}
Comenzamos el analisis empleando el circuito de la figura.

\sphinxincludegraphics[width=530\sphinxpxdimen,height=282\sphinxpxdimen]{{RLCpara}.png}

Calcularemos la transferencia del circuito. Vamos a emplear una fuente de corriente y calcular la tensión en el nodo comun.
\begin{equation*}
\begin{split}v_{g} = i_{g} \cdot \frac{1}{(\frac{1}{R} +\frac{1}{SL} + SC)}\end{split}
\end{equation*}
Calculando para \(S=j\omega\), para el analisis del comporatamiento del circuito en frecuencia:
\begin{equation*}
\begin{split}\frac{v_{g}}{i_{g}} =  \frac{1}{(\frac{1}{R} +\frac{1}{j\omega L} + j\omega C)}\end{split}
\end{equation*}\begin{equation*}
\begin{split}\frac{v_{g}}{i_{g}} =  \frac{j\omega }{C ( \frac{j \omega }{CR} +\frac{1}{LC} - \omega^2 )}\end{split}
\end{equation*}
Donde podemos normalizar la ecuación empleando los terminos \(Q\), ya presentado, y \(\omega_o^2 = \frac{1}{LC}\) como la frecuencia de resonancia.
\begin{equation*}
\begin{split}\frac{v_{g}}{i_{g}} =  \frac{j\omega }{C ( \frac{j \omega }{CR} + \omega_o^2 - \omega^2 )}\end{split}
\end{equation*}
Podemos remplazar el \(C = \frac{Q}{R \cdot \omega_o}\)
\begin{equation*}
\begin{split}\frac{v_{g}}{i_{g}} =  \frac{j  R \omega_o \omega}{Q(\frac{j   \omega_o \omega}{Q} +\omega_o^2  - \omega^2  )}\end{split}
\end{equation*}\begin{equation*}
\begin{split}\frac{v_{g}}{i_{g}} =  \frac{j  R \omega_o \omega}{(j \omega_o \omega +Q (\omega_o^2  - \omega^2  )}\end{split}
\end{equation*}
Sacando factor comun \(j \omega_o \omega\) y simplificando :
\begin{equation*}
\begin{split}\frac{v_{g}}{i_{g}} =  \frac{  R }{1 + j Q (\frac{\omega^2  - \omega_o^2}{\omega_o \omega})}\end{split}
\end{equation*}
Donde es facil reconocer el que el máximo de transferencia se produce cuando \(\omega^2 = \omega_o^2\) (resonancia).
\begin{equation*}
\begin{split}\frac{v_{g}}{i_{g}}(\omega_o) =  R\end{split}
\end{equation*}
El factor de selectividad relaciona el \(Q\) con el ancho de banda del circuito.

Para tener el ancho de banda, debemos buscar el ancho de banda donde la transferencia sea mayor a los \(3dB\).

El modulo de la transferencia,
\begin{equation*}
\begin{split}| \frac{v_{g}}{i_{g}} |=  \frac{  R }{\sqrt{1 +  (Q (\frac{f^2  - f_o^2}{f_o f}))^2}}\end{split}
\end{equation*}\begin{equation*}
\begin{split}\frac{R}{\sqrt{2}}=  \frac{  R }{\sqrt{1 +  (Q (\frac{f_c^2  - f_o^2}{f_o f_c}))^2}}\end{split}
\end{equation*}
Por lo tanto, las frecuencias donde cae \(3dB\).
\begin{equation*}
\begin{split}2 =  1 +  (Q (\frac{f_c^2  - f_o^2}{f_o f_c}))^2\end{split}
\end{equation*}\begin{equation*}
\begin{split}1 =   (Q (\frac{f_c^2  - f_o^2}{f_o f_c}))^2\end{split}
\end{equation*}\begin{equation*}
\begin{split}1 =   Q (\frac{f_c}{f_o}  - \frac{f_o}{f_c} )\end{split}
\end{equation*}\begin{equation*}
\begin{split}f_c =  - Q  f_o  + Q \frac{f_c^2}{f_o}\end{split}
\end{equation*}\begin{equation*}
\begin{split}f_c +   Q  f_o  - Q \frac{f_c^2}{f_o} = 0\end{split}
\end{equation*}\begin{equation*}
\begin{split}f_c^2  - f_c \frac{f_o}{Q}  -f_o^2  = 0\end{split}
\end{equation*}
Donde \(f_c\) puede tomtar los valores.
\begin{equation*}
\begin{split}f_c = \frac{f_o}{2 Q} (1 \pm \sqrt{4 Q^2 + 1})\end{split}
\end{equation*}\begin{equation*}
\begin{split}f_{c1} = \frac{f_o}{2 Q} (1 + \sqrt{4 Q^2 + 1})\end{split}
\end{equation*}\begin{equation*}
\begin{split}f_{c2} = \frac{f_o}{2 Q} (1 - \sqrt{4 Q^2 + 1})\end{split}
\end{equation*}
Entonces el ancho de banda:
\begin{equation*}
\begin{split}BW = f_{c1} -f_{c2} = \frac{f_o}{Q}\end{split}
\end{equation*}
donde \(f_o\) corresponde a la frecuencia de resonancia (\(\omega = 2 \pi f_o\)).

\sphinxincludegraphics[width=432\sphinxpxdimen,height=288\sphinxpxdimen]{{Qcon}.png}


\subsection{\protect\(Q_o\protect\) (\protect\(Q\protect\) libre) en inductores y capacitores}
\label{\detokenize{adaptacion/adaptacion:Q_o-(Q-libre)-en-inductores-y-capacitores}}
Los inductores y capaciores reales presentan perdidas. Esto quiere decir que a la frecuencia de trabajo, el comportamieto de estos componentes se pude modelizar (de la manera mas simple) como una inductancia o capacitancia, en paralelo con una resistencia de perdidas.

\sphinxincludegraphics[width=550\sphinxpxdimen,height=229\sphinxpxdimen]{{perdidaPara}.png}

El \(Q_o\) (libre) de un inductor para el modelo paralelo, dada una resistencia de perdida \(r_p\) se calcula como:
\begin{equation*}
\begin{split}Q_o =  \frac{r_p}{w_o \cdot L}\end{split}
\end{equation*}

\subsubsection{Inductores de alto Q para RF}
\label{\detokenize{adaptacion/adaptacion:Inductores-de-alto-Q-para-RF}}
Como ejemplo podemos ver como la curva de los inductores de alto Q que comercializa la empresa Johanson (\sphinxurl{https://www.johansontechnology.com/downloads/johanson-technology-rf-wirewound-chip-inductors.pdf})

\sphinxincludegraphics[width=910\sphinxpxdimen,height=348\sphinxpxdimen]{{QlibreL}.png}


\subsubsection{El factor de selectividad y la resistencia equivalente serie (ESR)}
\label{\detokenize{adaptacion/adaptacion:El-factor-de-selectividad-y-la-resistencia-equivalente-serie-(ESR)}}
Uno de los parámetros más importantes en la evaluación de un condensador de chip de alta frecuencia es el factor Q, o la resistencia en serie equivalente (ESR) relacionada.

Un condensador sin perdidas presenta un ESR de cero ohmios y sería puramente reactivo sin ningún componente real (resistiva). La corriente que pasa por el capacitor conduciría el voltaje a través exactamente 90 grados en todas las frecuencias.

Los capacitores no son ideales, y siempre exhibirá una cantidad finita de ESR. El ESR varía con la frecuencia de un capacitor dado y es “equivalente” porque su fuente proviene de las características de las estructuras de electrodo conductor y de la estructura dieléctrica aislante. Con el propósito de modelar, el ESR se representa como un elemento parásito de una sola serie. En las últimas décadas, todos los parámetros del condensador se midieron a un estándar de 1 MHz, pero en el mundo de alta
frecuencia actual, esto está lejos de ser suficiente. Los valores típicos para un buen condensador de alta frecuencia de un valor dado podrían funcionar en el orden de aproximadamente 0,05 ohmios a 200 MHz, 0,11 ohmios a 900 MHz y 0,14 ohmios a 2000 MHz.

El factor de calidad Q es un número adimensional que es igual a la reactancia del capacitor dividido por la resistencia parásita del capacitor (ESR). El valor de Q cambia mucho con la frecuencia, ya que tanto la reactancia como la resistencia cambian con la frecuencia. La reactancia de un condensador cambia enormemente con la frecuencia o con el valor de capacitancia y, por lo tanto, el valor Q podría variar en gran medida.

\sphinxurl{http://www.vishay.com/docs/28534/highqdielectric.pdf}

\sphinxincludegraphics[width=1084\sphinxpxdimen,height=260\sphinxpxdimen]{{QlibreC}.png}


\subsection{\protect\(Q_c\protect\) (\protect\(Q\protect\) cargado) en circuitos RLC paralelos.}
\label{\detokenize{adaptacion/adaptacion:Q_c-(Q-cargado)-en-circuitos-RLC-paralelos.}}
El \(Q_c\) nos permite conocer el comporatamiento del circuito cuando esta cargado por la impedancia de la fuente y la de la carga.

\sphinxincludegraphics[width=825\sphinxpxdimen,height=272\sphinxpxdimen]{{maxPot}.png}

Del circuito resonante paralelo, en resonancia (donde se anula la componente imaginaria) la resistencia total \(\frac{1}{r} = \frac{1}{r_p} + \frac{1}{R_{ext}}\) se calcula como:

donde \(R_{ext} = \frac{r_g R_L}{r_g + R_L}\)
\begin{equation*}
\begin{split}Q_c =  \frac{r}{w_o \cdot L} =  r  \cdot w_o \cdot C\end{split}
\end{equation*}
Entonces, de igual manera multiplicando ambos terminos por \(w_o \cdot L\):
\begin{equation*}
\begin{split}\frac{w_o \cdot L}{r} = \frac{w_o \cdot L}{r_p} + \frac{w_o \cdot L}{R_{ext}}\end{split}
\end{equation*}\begin{equation*}
\begin{split}\frac{1}{Q_c} =  \frac{1}{Q_o} + \frac{w_o \cdot L}{R_{ext}}\end{split}
\end{equation*}
\sphinxurl{https://www.coilcraft.com/pdfs/Doc945\_Inductors\_as\_RF\_Chokes.pdf}

\sphinxurl{http://www.ee.iitm.ac.in/~ani/2011/ee6240/pdf/AN721\_AppNote\_Matching.pdf}

\sphinxurl{https://www.spelektroniikka.fi/kuvat/schot7.pdf}


\subsection{Circuito RLC serie.}
\label{\detokenize{adaptacion/adaptacion:Circuito-RLC-serie.}}
\sphinxincludegraphics[width=410\sphinxpxdimen,height=210\sphinxpxdimen]{{Qcserie2}.png}

En un circuito RLC serie en resonancia,
\begin{equation*}
\begin{split}w_o  = \frac{1}{ \sqrt{LC} }\end{split}
\end{equation*}
,


\subsubsection{\protect\(Q_c\protect\) (\protect\(Q\protect\) cargado) en circuitos RLC serie.}
\label{\detokenize{adaptacion/adaptacion:Q_c-(Q-cargado)-en-circuitos-RLC-serie.}}
El \(Q_o\) (libre) de un inductor para el modelo paralelo, dada una resistencia de perdida \(r_p\) se calcula como:
\begin{equation*}
\begin{split}Q_o =  \frac{w_o \cdot L}{r_p}\end{split}
\end{equation*}
El \(Q_c\) (cargado) de este circuito resonante para el modelo serie, dada una resistencia total \(r = r_p + R_{ext}\) se calcula como \(Q_c = \frac{w_o \cdot L}{r}\), donde \(R_{ext} = r_g + R\).

Entonces, dividiendo ambos miembros de la ecuación por \(w_o \cdot L\):
\begin{equation*}
\begin{split}\frac{r}{w_o \cdot L} = \frac{r_s}{w_o \cdot L} + \frac{R_{ext}}{w_o \cdot L}\end{split}
\end{equation*}\begin{equation*}
\begin{split}\frac{1}{Q_c} =  \frac{1}{Q_o} + \frac{R_{ext}}{w_o \cdot L}\end{split}
\end{equation*}

\chapter{Conversión serie a paralelo}
\label{\detokenize{adaptacion/adaptacion:Conversi_xf3n-serie-a-paralelo}}
Buscaremos la relación entre un circuito resonante serie y un resonante paralelo. Esto será muy util para el diseño y verificación de los filtros, ya que no permitiran agilizar los calculos.

\sphinxincludegraphics[width=791\sphinxpxdimen,height=335\sphinxpxdimen]{{paraserie}.png}

En un circuito resonante paralelo, la impedancia de entrada se calcula como:
\begin{equation*}
\begin{split}\frac{1}{ Z_p}  = \frac{1}{R_p}   + \frac{1}{j \cdot X_p}\end{split}
\end{equation*}
Separando la parte real de la parte imaginaria:
\begin{equation*}
\begin{split}\frac{1}{ Z_p}  = \frac{R_p + j \cdot X_p }{R_p \cdot j \cdot X_p }\end{split}
\end{equation*}\begin{equation*}
\begin{split}Z_p  =  \frac{R_p \cdot j \cdot X_p }{R_p + j \cdot X_p } \cdot \frac{R_p - j \cdot X_p }{R_p - j \cdot X_p }\end{split}
\end{equation*}\begin{equation*}
\begin{split}Z_p  =  \frac{R_p \cdot j \cdot X_p \cdot (R_p - j \cdot X_p) }{(R_p^2 +  X_p^2 ) }\end{split}
\end{equation*}\begin{equation*}
\begin{split}Z_p  =  \frac{( R_p \cdot X_p^2 ) + j \cdot ( R_p^2 \cdot X_p) }{(R_p^2 +   X_p^2) }\end{split}
\end{equation*}\begin{equation*}
\begin{split}Z_p  =  \frac{( R_p \cdot X_p^2 ) }{(R_p^2 +   X_p^2) } + j \cdot \frac{ ( R_p^2 \cdot X_p) }{(R_p^2 +   X_p^2)}\end{split}
\end{equation*}\begin{equation*}
\begin{split}Z_p  =  \frac{( R_p \cdot (\frac{R_p}{Q_o})^2 ) }{(R_p^2 +   (\frac{R_p}{Q_o})^2) } + j \cdot \frac{ ( R_p^2 \cdot X_p) }{(R_p^2 +   (\frac{R_p}{Q_o})^2)}\end{split}
\end{equation*}
En un circuito resonante serie, la impedancia de entrada se calcula como:
\begin{equation*}
\begin{split}Z_s  = R_s   + j \cdot X_s\end{split}
\end{equation*}
Entonces, en resonancia, igualando la parte real de la impedancia :
\begin{equation*}
\begin{split}R_s = \frac{( R_p \cdot X_p^2 ) }{(R_p^2 +   X_p^2) }\end{split}
\end{equation*}
En resonancia, \(Q_o = \frac{R}{X_p}\), entonces \(X_p = \frac{R}{Q_o}\). Remplazando resulta:
\begin{equation*}
\begin{split}R_s = \frac{( R_p \cdot (\frac{R_p}{Q_o})^2 ) }{(R_p^2 +   (\frac{R_p}{Q_o})^2) }\end{split}
\end{equation*}
Sacando \(R_p^2\) como factor comun y simplificando resulta:
\begin{equation*}
\begin{split}R_s = \frac{ R_p \cdot (\frac{1}{Q_o^2})  }{(1 +   (\frac{1}{Q_o^2})) } = \frac{ R_p }{(1 +   Q_o^2) }\end{split}
\end{equation*}
Por lo tanto, en resonancia, igualando la parte imaginaria de la impedancia :
\begin{equation*}
\begin{split}X_s = \frac{ ( R_p^2 \cdot X_p) }{(R_p^2 +   (\frac{R_p}{Q_o})^2)}\end{split}
\end{equation*}\begin{equation*}
\begin{split}X_s = \frac{ X_p }{(1 +   \frac{1}{Q_o^2})}\end{split}
\end{equation*}
De las ecuaciones, podemos concliuir que para realizar una conversion de serie a paralelo, donde el \(Q_o\) se suele llamar \(Q_m\) (Q de adaptación o ‘matching’), ya que no solo se emplea para las perdidas de los compoentes:
\begin{equation*}
\begin{split}R_s =  \frac{ R_p }{(1 +   Q_m^2) }\end{split}
\end{equation*}\begin{equation*}
\begin{split}X_s = \frac{ X_p }{(1 +   \frac{1}{Q_m^2})}\end{split}
\end{equation*}
Si el \(Q_m\) es mayor a 10 podemos despreciar el termino \(\frac{1}{Q_m^2}\)
\begin{equation*}
\begin{split}X_s \sim X_p\end{split}
\end{equation*}

\section{Ejemplo conversión serie a paralelo}
\label{\detokenize{adaptacion/adaptacion:Ejemplo-conversi_xf3n-serie-a-paralelo}}
Se desea diseñar una red de adaptación para transformar una carga de \(R_L = 50 \Omega\) para que presente a el colector de un transistor una resistencia de \(R_L' = 1000 \Omega\) a la frecuencia de \(f_o = 2 MHz\). Por simplicidad, suponenemos que todos los componentes no tienen perdidas (son ideales) y que la salida del transistor presenta una admitancia resistiva pura.

Para resolver este diseño, debemos primero realizar la conversión de serie a paralelo del inductor y la resistencia. Esta conversión tiene que darnos como resultado una resistencia paralelo de \(R_L' = 1000 \Omega\). La conversión de serie a paralelo depende del valor de \(Q_m\) (\(Q\) de matching).
\begin{equation*}
\begin{split}R_p =   R_s  (1 +   Q_m^2)\end{split}
\end{equation*}\begin{equation*}
\begin{split}Q_m =   \sqrt{\frac{R_p}{R_s} -1}\end{split}
\end{equation*}\begin{equation*}
\begin{split}Q_m =   \sqrt{\frac{1000}{50} -1}= 4.36\end{split}
\end{equation*}
Entonces con un \(Q_m = 4.36\) la resistencia del circuito paralelo se comporta como una resistencia de \(R_p = 1000 \Omega\).

Debemos conocer el valor del inductor que permite tener un \(Q_m = 4.36\). Dado que el valor que queremos conocer corresponde al inductor \(L\) serie, empleamos el calculo del \(Q_m\) del circuito serie (recordando que la correinte es el parametro comun para el calculo de la potencia en ambos componentes).
\begin{equation*}
\begin{split}Q_m =  \frac{i^2 X_L}{i^2 r_s}\end{split}
\end{equation*}\begin{equation*}
\begin{split}Q_m =  \frac{X_L}{r_s}\end{split}
\end{equation*}\begin{equation*}
\begin{split}X_L = Qm r_s = 4.36 \cdot 50 \Omega = 217.94 \Omega\end{split}
\end{equation*}
siento \(X_L = w_o \cdot L\), donde \(w_o = 2 \pi f_o\). Entonces,
\begin{equation*}
\begin{split}L = \frac{X_L}{2 \pi f_o}\end{split}
\end{equation*}\begin{equation*}
\begin{split}L = \frac{217.94 \Omega}{2 \pi 2\times 10^6 Hz} = 17.3 \mu Hz\end{split}
\end{equation*}
Para obtener el valor de \(C\), necesitamos conocer el valor del inductor correspondiente al circuito paralelo. Debemos calcular su valor para el circuito paralelo.
\begin{equation*}
\begin{split}X_p =  X_s (1 +   \frac{1}{Q_m^2})\end{split}
\end{equation*}\begin{equation*}
\begin{split}X_L' =  X_L (1 +   \frac{1}{Q_m^2})\end{split}
\end{equation*}\begin{equation*}
\begin{split}X_L' =  217.94 \Omega (1 +   \frac{1}{4.36^2})= 229.41 \Omega\end{split}
\end{equation*}
El valor que resuena con el capacitor entonces es \(X_C = X_L'\), siendo \(X_C = \frac{1}{w_o \cdot C}\). Hacemos entonces el calculo del valor del capacitor \(C\).
\begin{equation*}
\begin{split}C = \frac{1}{w_o \cdot X_C}\end{split}
\end{equation*}\begin{equation*}
\begin{split}C = \frac{1}{2 \pi 2\times 10^6 Hz \cdot 229.41 \Omega} = 346.8 pF\end{split}
\end{equation*}
\sphinxincludegraphics[width=432\sphinxpxdimen,height=288\sphinxpxdimen]{{converSerPAr}.png}

Como vemos en este ejemplo, se logra la adaptación de la resistencia mediante la conversión de serie a paralelo. Dado que se emplean solo dos componentes reactivos, el \(Q_c\) queda impuesto por el circuito y no puede ser modificado sin afectar la adaptación.


\chapter{Máxima transferencia de energía a \protect\(Q\protect\) constante.}
\label{\detokenize{adaptacion/adaptacion:M_xe1xima-transferencia-de-energ_xeda-a-Q-constante.}}
Se desea encontrar el valor de \(R_L\) que maximice la tranferencia de energía desde el generado a la carga, teniendo en cuenta que se emplea un circuito sintonizado \(LC\) con perdidas (\(r_p\)) donde se busca el valor de \(R_L\) para lograr la máxima transferencia de energía desde la fuente para un dado un \(Q_c\).

Dado que el circuito sintonizado esta compuesto por dos componentes, \(L\) y \(C\), para cada valor de \(R_L\) podriamos proponer que el inductor \(L\) se escoja para tener un \(Q_c\) dado y, luego, se buscará el valor de \(C\) que sintonice a \(w_o\). Asumimos que el \(Q_o\) libre del inductor no varia en este analisis.

\sphinxincludegraphics[width=825\sphinxpxdimen,height=272\sphinxpxdimen]{{maxPot}.png}

La potencia sobre \(R_L\)
\begin{equation*}
\begin{split}P_{R_L} = \frac{v_L^2}{ R_L}\end{split}
\end{equation*}
La tensión \(v_L\), que se expresa en valores eficaces, se puede obtener en función de la fuente de corriente que alimenta al circuito como:
\begin{equation*}
\begin{split}v_L = i_g \cdot \frac{1}{\frac{1}{r_g} + \frac{1}{r_p} + \frac{1}{R_L} }\end{split}
\end{equation*}
Remplazando el valor de \(v_L\) en \(P_{R_L}\):
\begin{equation*}
\begin{split}P_{R_L} = \frac{(i_g \cdot \frac{1}{\frac{1}{r_g} + \frac{1}{r_p} + \frac{1}{R_L} })^2}{ R_L}\end{split}
\end{equation*}
Operando,
\begin{equation*}
\begin{split}P_{R_L} = \frac{i_g^2}{  R_L \cdot (\frac{1}{r_g} + \frac{1}{r_p} + \frac{1}{R_L})^2 }\end{split}
\end{equation*}
De la ecuación, \(r_p\) buscamos escribir en función de \(Q_c\).

Sabemos que \(r_p = w_o \cdot L \cdot Q_o\).
\begin{equation*}
\begin{split}\frac{1}{Q_c} =  \frac{1}{Q_o} + \frac{w_o \cdot L}{R }\end{split}
\end{equation*}\begin{equation*}
\begin{split}w_o \cdot L = R \cdot (\frac{1}{Q_c} -  \frac{1}{Q_o} )\end{split}
\end{equation*}
donde \(R = \frac{R_L \cdot r_g }{ R_L + r_g }\).
\begin{equation*}
\begin{split}Q_o = \frac{r_p}{ w_o \cdot L}\end{split}
\end{equation*}\begin{equation*}
\begin{split}r_p = R \cdot (\frac{1}{Q_c} -  \frac{1}{Q_o} ) \cdot Q_o\end{split}
\end{equation*}\begin{equation*}
\begin{split}r_p = \frac{R_L \cdot r_g }{ R_L + r_g  } \cdot (\frac{Q_o}{Q_c} -  1 )\end{split}
\end{equation*}
Remplazando \(r_p\) en la ecuación de la potencia \(P_{R_L}\):
\begin{equation*}
\begin{split}P_{R_L} = \frac{i_g^2}{  R_L \cdot (\frac{1}{r_g} + \frac{1}{\frac{R_L \cdot r_g }{ R_L + r_g  } \cdot (\frac{Q_o}{Q_c} -  1 ) } + \frac{1}{R_L})^2 }\end{split}
\end{equation*}\begin{equation*}
\begin{split}P_{R_L} = \frac{i_g^2}{  R_L \cdot (\frac{1}{r_g} + \frac{R_L + r_g  }{R_L \cdot r_g \cdot (\frac{Q_o}{Q_c} -  1 ) } + \frac{1}{R_L})^2 }\end{split}
\end{equation*}\begin{equation*}
\begin{split}P_{R_L} = \frac{i_g^2 \cdot R_L^2 \cdot r_g^2 }{  R_L \cdot (r_g + \frac{R_L + r_g  }{ (\frac{Q_o}{Q_c} -  1 ) } + R_L)^2 }\end{split}
\end{equation*}\begin{equation*}
\begin{split}P_{R_L} = \frac{i_g^2 \cdot R_L \cdot r_g^2 }{  (r_g + \frac{R_L + r_g  }{ (\frac{Q_o}{Q_c} -  1 ) } + R_L)^2 }\end{split}
\end{equation*}\begin{equation*}
\begin{split}P_{R_L} = \frac{i_g^2 \cdot R_L \cdot r_g^2 }{ (R_L + r_g)^2 \cdot (1 + \frac{ 1 }{ (\frac{Q_o}{Q_c} -  1 ) } )^2 }\end{split}
\end{equation*}\begin{equation*}
\begin{split}P_{R_L} = \frac{i_g^2 \cdot R_L \cdot r_g^2 }{ (R_L + r_g)^2 \cdot ( \frac{ \frac{Q_o}{Q_c} }{ (\frac{Q_o}{Q_c} -  1 ) } )^2 }\end{split}
\end{equation*}\begin{equation*}
\begin{split}P_{R_L} = \frac{i_g^2 \cdot R_L \cdot r_g^2 }{ (R_L + r_g)^2 \cdot ( \frac{ Q_o }{ Q_o -  Q_c  } )^2 }\end{split}
\end{equation*}\begin{equation*}
\begin{split}P_{R_L} = \frac{i_g^2 \cdot R_L }{ (\frac{R_L + r_g}{r_g}  \cdot  \frac{ Q_o }{ Q_o -  Q_c  } )^2 }\end{split}
\end{equation*}\begin{equation*}
\begin{split}P_{R_L} =  \frac{i_g^2}{(\frac{ Q_o }{ Q_o -  Q_c  })^2}   \cdot \frac{ R_L \cdot  r_g^2}{ (R_L + r_g)^2 }\end{split}
\end{equation*}
Buscando el máximo valor de \(P_{R_L}\):
\begin{equation*}
\begin{split}\frac{\partial P_{R_L}}{\partial R_L } = \frac{i_g^2}{(\frac{ Q_o }{ Q_o -  Q_c  })^2}   \cdot \frac{[(r_g^2 \cdot (R_L^2 + r_g)^2) - 2 \cdot (R_L +  r_g) \cdot (R_L \cdot  r_g^2)  ]}{(R_L +  r_g)^4} = 0\end{split}
\end{equation*}\begin{equation*}
\begin{split}(r_g^2 \cdot (R_L + r_g)^2) - 2 \cdot (R_L +  r_g) \cdot (R_L \cdot  r_g^2)   = 0\end{split}
\end{equation*}\begin{equation*}
\begin{split}r_g^2 \cdot (R_L + r_g)^2 = 2 \cdot (R_L +  r_g) \cdot (R_L \cdot  r_g^2)\end{split}
\end{equation*}\begin{equation*}
\begin{split}(R_L + r_g)  = 2 \cdot R_L\end{split}
\end{equation*}\begin{equation*}
\begin{split}R_L = r_g\end{split}
\end{equation*}
Entonces :
\begin{equation*}
\begin{split}P_{R_L} =  \frac{i_g^2}{(\frac{ Q_o }{ Q_o -  Q_c  })^2}   \cdot \frac{ r_g \cdot  r_g^2}{ (r_g + r_g)^2 }\end{split}
\end{equation*}\begin{equation*}
\begin{split}P_{R_L} =  \frac{i_g^2}{(\frac{ Q_o }{ Q_o -  Q_c  })^2}   \cdot \frac{ r_g }{ 4 }\end{split}
\end{equation*}\begin{equation*}
\begin{split}P_{R_L MAX: (RL =  rg) } =  \frac{i_g^2 \cdot r_g}{4}   \cdot (1 - \frac{ Q_c }{ Q_o })^2\end{split}
\end{equation*}
donde \(i_g\) es valor RMS. Si se emplea valores pico:
\begin{equation*}
\begin{split}P_{R_L MAX: (RL =  rg) } =  \frac{i_g^2 \cdot r_g}{8}   \cdot (1 - \frac{ Q_c }{ Q_o })^2\end{split}
\end{equation*}
donde reconocemos el termino \(\frac{i_g^2 \cdot r_g}{8}\) como \(P_{av}\) o potencia disponible.
\begin{equation*}
\begin{split}P_{R_L MAX: (RL =  rg) } =  P_{av}   \cdot (1 - \frac{ Q_c }{ Q_o })^2\end{split}
\end{equation*}

\section{Ejemplo}
\label{\detokenize{adaptacion/adaptacion:Ejemplo}}
\sphinxincludegraphics[width=825\sphinxpxdimen,height=272\sphinxpxdimen]{{maxPot}.png}

Suponer que \(r_g = 10 \Omega\), \(f_o = 1 MHz\), \(Q_o = 100\) y \(Q_c = 10\).

¿Cual debe ser el valor de \(R_L\) para obtener la maxima transferencia de energía, si se desae tener un \(Q_c = 10\)?

Por la demostración realizada, la resistencia debe ser \(R_L = r_g\), sin importar el valor de \(r_p\).

Por lo tanto, basado en el teorema de máxima transferencia de energía a \(Q_c\) constante:
\begin{equation*}
\begin{split}R_L = r_g = 10 \Omega\end{split}
\end{equation*}
Para este ejemplo, el valor del inductor entonces:
\begin{equation*}
\begin{split}\frac{1}{Q_c} = \frac{1}{Q_o} + \frac{w_o L}{R_{ext}}\end{split}
\end{equation*}
Donde corresponde a todas las resistencias externas al inductor y su respectiva resistencia de perdidas \(r_p\). En este caso, \(R_{ext} = \frac{r_g R_L}{r_g R_L} = 5 \Omega\)
\begin{equation*}
\begin{split}\frac{1}{Q_c} = \frac{1}{Q_o} + \frac{w_o L}{R_{ext}}\end{split}
\end{equation*}\begin{equation*}
\begin{split}L= \frac{R_{ext}(\frac{1}{Q_c} - \frac{1}{Q_o})}{w_o}\end{split}
\end{equation*}\begin{equation*}
\begin{split}L= \frac{5 \Omega (\frac{1}{10} - \frac{1}{30})}{2 \pi 1 MHz} = 71 nHy\end{split}
\end{equation*}
En este ejemplo, el valor de la resistencia de perdidas del adaptador (que asociamos al inductor):
\begin{equation*}
\begin{split}r_p = \frac{Q_o}{w_o L} = 222 \Omega\end{split}
\end{equation*}
El valor del capacitor \(C\) es aquel que sintoniza a \(L\).
\begin{equation*}
\begin{split}w_o^2 = \frac{1}{C L}\end{split}
\end{equation*}\begin{equation*}
\begin{split}C = \frac{1}{L w_o^2}\end{split}
\end{equation*}\begin{equation*}
\begin{split}C = \frac{1}{53 nHy  (2 \pi 1 MHz)^2} = 353.67 pF\end{split}
\end{equation*}
Nos queda conocer las perdidas del adaptador. Para ello, dado la ecuación del calculo de la potencia:
\begin{equation*}
\begin{split}P_{R_L MAX: (RL =  rg) } =  P_{av}   \cdot (1 - \frac{ Q_c }{ Q_o })^2\end{split}
\end{equation*}
Las perdidas del adaptador se calcula como:
\begin{equation*}
\begin{split}Perdidas = (1 - \frac{ Q_c }{ Q_o })^2 = (1 - \frac{ 10 }{ 100 })^2 = 0.81\end{split}
\end{equation*}
\sphinxincludegraphics[width=720\sphinxpxdimen,height=720\sphinxpxdimen]{{ejemploMTEQcte}.png}


\chapter{Redes de adaptación, circuitos resonantes con derivación.}
\label{\detokenize{adaptacion/adaptacion:Redes-de-adaptaci_xf3n,-circuitos-resonantes-con-derivaci_xf3n.}}

\section{El adaptador tipo “L”}
\label{\detokenize{adaptacion/adaptacion:El-adaptador-tipo-_u201cL_u201d}}
\sphinxincludegraphics[width=977\sphinxpxdimen,height=200\sphinxpxdimen]{{Impedance_matching_general}.png}

El problema general se ilustra en la siguiente figura: un generador con un impedano interno 𝑍𝑆 suministra energía a una carga pasiva 𝑍𝐿, a través de una red de coincidencia de 2 puertos.

Este problema se denomina comúnmente “el problema de doble coincidencia”. La coincidencia de impedancia es importante por las siguientes razones:
\begin{itemize}
\item {} 
Maximizando la transferencia de potencia. La potencia máxima se entrega a la carga cuando el generador y la carga coinciden con la línea y la pérdida de potencia en la línea se minimiza.

\item {} 
Mejora de la relación señal / ruido del sistema.

\item {} 
Reducción de errores de amplitud y fase.

\item {} 
Reducción de la potencia reflejada hacia el generador.

\end{itemize}

Mientras la impedancia de carga 𝑍𝐿 tenga una parte positiva real, siempre se puede encontrar una red coincidente. Hay muchas opciones disponibles y los ejemplos a continuación solo describen algunas. Los ejemplos están tomados del libro de D.Pozar “Ingeniería de microondas”, cuarta edición. “Microwave Engineering”, 4th edition.


\section{Adaptador con elementos de constantes concentradas.}
\label{\detokenize{adaptacion/adaptacion:Adaptador-con-elementos-de-constantes-concentradas.}}
Para comenzar, supongamos que la red adaptadora que no tiene pérdidas y la impedancia característica de la línea de alimentación es \(Z_o\):

\sphinxincludegraphics[width=587\sphinxpxdimen,height=167\sphinxpxdimen]{{Impedance_matching_lumped1}.png}

El tipo más simple de red es la red “L”, que utiliza dos elementos reactivos para adaptadar una impedancia de carga arbitraria. Existen dos configuraciones posibles y se ilustran en las siguientes figuras. En cualquiera de las configuraciones, los elementos reactivos pueden ser inductivos de capacitivos, dependiendo de la impedancia de carga.

\sphinxincludegraphics[width=476\sphinxpxdimen,height=455\sphinxpxdimen]{{generico}.png}

Supongamos que la carga es \(Z_L = 200 \Omega\) para una línea \(Z_o = 100 \Omega\) a la frecuencia de \(f_o = 500 MHz\).

El adaptador tipo “L” debe llevar de \(\Re{(Z_L)} = 200 \Omega\) a la impedancia de \(Z_o = 100 \Omega\) de la linea. En este esquema, deseamos reducir el valor de la resistencia, por lo tanto empleamos una conversión de paralelo a serie. Buscamos una red L\sphinxhyphen{}C que corresponda a la primer propuesta. Tenemos dos alternativas, las cuales se ilustran en la sigueinte figura.

\sphinxincludegraphics[width=547\sphinxpxdimen,height=477\sphinxpxdimen]{{ejemplo_LCa}.png}

Esta conversión tiene que darnos como resultado una resistencia serie de \(R_L' = 100 \Omega\). La conversión de serie a paralelo depende del valor de \(Q_m\) (\(Q\) de matching).
\begin{equation*}
\begin{split}R_s =   \frac{R_p}{(1 +   Q_m^2)}\end{split}
\end{equation*}\begin{equation*}
\begin{split}Q_m =   \sqrt{\frac{200 \Omega}{100 \Omega} -1} = 1\end{split}
\end{equation*}
El \(Q_m = 1\) permite que la resistencia del circuito serie se comporta como una resistencia de \(R_s = 100 \Omega\).

Calculemos ahora la susceptancia \(B_p\) para el \(Q_m\) necesario. Siendo este un circuito paralelo que tiene como parametro común la tensión:
\begin{equation*}
\begin{split}Q_m = (v_p^2 B_p) \cdot  \frac{R_s}{v_p^2} = R_s \cdot B_p\end{split}
\end{equation*}\begin{equation*}
\begin{split}B_p = \frac{Q_m}{R_p} = \frac{1}{200 \Omega} = 5 mS\end{split}
\end{equation*}
Para conocer el valor de la reactancia \(X_s\), debemos realizar la conversión paralelo a serie de la susceptancia \(B_p\).
\begin{equation*}
\begin{split}B_p' = B_p \cdot (1 + \frac{1}{Q_m^2})\end{split}
\end{equation*}\begin{equation*}
\begin{split}B_p' = 5 mS (1 + \frac{1}{1^2}) = 10 mS\end{split}
\end{equation*}
El valor de \(X_s\) que resuena con \(B_p'\).
\begin{equation*}
\begin{split}X_s = \frac{1}{B_p'} = \frac{1}{10 mS} = 100 \Omega\end{split}
\end{equation*}
\sphinxincludegraphics[width=576\sphinxpxdimen,height=236\sphinxpxdimen]{{ejemplo_LCb}.png}

\sphinxstylestrong{Primera alternativa: C serie, L derivación}

\sphinxincludegraphics[width=926\sphinxpxdimen,height=412\sphinxpxdimen]{{ejemplo_adapL1}.png}
\begin{equation*}
\begin{split}B_p = \frac{1}{2 \pi f_o L}\end{split}
\end{equation*}\begin{equation*}
\begin{split}L = \frac{1}{2 \pi 500 MHz  5 mS} = 63.66 nHy\end{split}
\end{equation*}\begin{equation*}
\begin{split}X_s = \frac{1}{2 \pi f_o C}\end{split}
\end{equation*}\begin{equation*}
\begin{split}C = \frac{1}{2 \pi 500 MHz  400 \Omega} = 3.18 pF\end{split}
\end{equation*}
\sphinxincludegraphics[width=432\sphinxpxdimen,height=288\sphinxpxdimen]{{ejemplo2a}.png}

\sphinxincludegraphics[width=432\sphinxpxdimen,height=288\sphinxpxdimen]{{ejemplo2asmitha}.png}

\sphinxstylestrong{Segunda alternativa: L serie, C derivación} \sphinxincludegraphics[width=930\sphinxpxdimen,height=364\sphinxpxdimen]{{ejemplo_adapL2}.png}
\begin{equation*}
\begin{split}B_p = 2 \pi f_o C\end{split}
\end{equation*}\begin{equation*}
\begin{split}C = \frac{B_p}{2 \pi 500 MHz} = 1.59 pF\end{split}
\end{equation*}\begin{equation*}
\begin{split}X_s = 2 \pi f_o L\end{split}
\end{equation*}\begin{equation*}
\begin{split}L = \frac{400 \Omega}{2 \pi 500 MHz} = 31.8 nHy\end{split}
\end{equation*}
\sphinxincludegraphics[width=640\sphinxpxdimen,height=480\sphinxpxdimen]{{ejemplo2b}.png}

\sphinxincludegraphics[width=432\sphinxpxdimen,height=288\sphinxpxdimen]{{ejemplo2bsmitha}.png}

\sphinxstylestrong{Conclusiones}

Ambas alternativas cumplen con el objetivo de adaptar la impedancia de carga a la linea. Podemos adaptar a una unica frecuencia. Para el caso de máxima transferencia de energía, el \(Q_c\) del circuito queda impuesto. Si se desea ademas un valor determiando, se necesitaran mas componentes.

Por otro lado, podemos ver que la primer alternativa corresponde a un pasa alto y la segunda a un pasabajos. La elección dependerá del uso de este circuito.


\subsection{Ejemplo adaptador tipo L, adaptador para antena de 11m.}
\label{\detokenize{adaptacion/adaptacion:Ejemplo-adaptador-tipo-L,-adaptador-para-antena-de-11m.}}
Supóngase querer adaptar una antena látigo de \(3 m\) que presenta \(50 \Omega\) a aprox. \(27MHz\), pero que va a usarse a \(3 MHz\), presentando en este caso una impedancia \(Z_{ant}= 0.3 \Omega + j \frac{1}{2 \pi 3MHz 30nF}\).

\sphinxincludegraphics[width=552\sphinxpxdimen,height=256\sphinxpxdimen]{{ejemplo1a}.png}

Se desea que el equipo transmisor, la fuente de corriente, tenga una carga de \(Z_{in} = 50\Omega+j0\Omega\) @ \(3 MHz\) para que esté adaptada, se recurre así al circuito de adaptación mostrado en la figura compuesto por L y C. Suponga que la resistencia de perdida de L es \(r_p = 1\Omega\). La potencia disponible del transmisor es \(P_{disp} = 100 W\).

\sphinxstylestrong{Calculo L y C sin perdidas}

Para el cálculo del inductor primero debemos neutralizar el efecto de la capacitancia producida en la antena. Para ello, separamos el inductor en dos inductores conectados en serie, como muestra la figura.

\sphinxincludegraphics[width=384\sphinxpxdimen,height=182\sphinxpxdimen]{{ejemplo1b}.png}

Calculamos \(L_b\) para que neutralice o resuene con la capacidad.
\begin{equation*}
\begin{split}X_c =\frac{1}{2 \pi 2MHz 30pF} = 2652.5 \Omega\end{split}
\end{equation*}\begin{equation*}
\begin{split}L_b = \frac{X_c}{2 \pi 2MHz} = 211 \mu Hy\end{split}
\end{equation*}
El circuito resultante en la rama es un RL serie en paralelo con el capacitor de adaptación. Es necesario que el circuito serie RL se presente como un circuito paralelo RL, donde R debe tomar valor el valor de \(50 \Omega\). Para ello necesitamos calcular el \(Q_M\) que permita obtener el valor buscado.
\begin{equation*}
\begin{split}R_p = R_s (1+Q_M^2)\end{split}
\end{equation*}\begin{equation*}
\begin{split}Q_M = \sqrt[]{\frac{R_p}{R_s}-1} = \sqrt[]{\frac{50}{1}-1} = 7\end{split}
\end{equation*}
A partir de \(Q_M\) calculamos \(L_a\).
\begin{equation*}
\begin{split}Q_M = \frac{\omega L_a}{R_s}\end{split}
\end{equation*}\begin{equation*}
\begin{split}L_a = 557nHy\end{split}
\end{equation*}
Para finalmente obtener el valor de \(L\).
\begin{equation*}
\begin{split}L = L_a + L_b =  211.557 \mu Hy\end{split}
\end{equation*}
El capacitor que resuena con el \(L'\) equivalente paralelo entonces,
\begin{equation*}
\begin{split}C  = 11.14 nF\end{split}
\end{equation*}\begin{equation*}
\begin{split}Q_o = \frac{\omega L}{r_p} = 7\end{split}
\end{equation*}\begin{equation*}
\begin{split}\frac{1}{Q_c} = \frac{1}{Q_o} + \frac{\omega L_a'}{R_{ext}}\end{split}
\end{equation*}\begin{equation*}
\begin{split}\frac{1}{Q_c} = \frac{1}{Q_o} + \frac{2 \pi 2MHz 568.3 nHy }{25}\end{split}
\end{equation*}\begin{equation*}
\begin{split}Q_c= 2.33\end{split}
\end{equation*}
\sphinxincludegraphics[width=421\sphinxpxdimen,height=189\sphinxpxdimen]{{ejemplo1c}.png}

\sphinxstylestrong{Calcular la potencia en la carga.}

Para el calculo de la potencia, en base a la potencia disponible calculamos la carga.
\begin{equation*}
\begin{split}P_{disp} = \frac{i_g^2 r_g}{8} = 100 W\end{split}
\end{equation*}\begin{equation*}
\begin{split}i_g = \sqrt{\frac{P_{disp} 8}{r_g}}\end{split}
\end{equation*}\begin{equation*}
\begin{split}i_g = \sqrt{\frac{50 \cdot 8}{50 \Omega}}= 4 A\end{split}
\end{equation*}

\subsubsection{Gráficos}
\label{\detokenize{adaptacion/adaptacion:Gr_xe1ficos}}
\sphinxincludegraphics[width=432\sphinxpxdimen,height=288\sphinxpxdimen]{{ejemploAntenaZia}.png}

\sphinxincludegraphics[width=432\sphinxpxdimen,height=288\sphinxpxdimen]{{ejemploAntenaZib}.png}

\sphinxincludegraphics[width=432\sphinxpxdimen,height=288\sphinxpxdimen]{{ejemploAntenaPinA}.png}

\sphinxincludegraphics[width=432\sphinxpxdimen,height=288\sphinxpxdimen]{{ejemploAntenaGPA}.png}

\sphinxincludegraphics[width=432\sphinxpxdimen,height=288\sphinxpxdimen]{{ejemploAntenaPinB}.png}

\sphinxincludegraphics[width=432\sphinxpxdimen,height=288\sphinxpxdimen]{{ejemploAntenaGPB}.png}


\section{Transformador}
\label{\detokenize{adaptacion/adaptacion:Transformador}}

\subsection{Transformadores de RF}
\label{\detokenize{adaptacion/adaptacion:Transformadores-de-RF}}
Los transformadores de RF son principalmente utilizados en circuitos para: 1. Adaptación de impedancia para lograr la máxima transferencia de potencia y para suprimir la reflexión de señal no deseada. 2. Voltaje, corriente ascendente o descendente. 3. Aislamiento de CC entre circuitos al tiempo que permite una transmisión de CA eficiente. 4. Interfaz entre circuitos balanceados y no balanceados; ejemplo: amplificadores balanceados.


\subsubsection{CIRCUITOS DE TRANSFORMADORES Y RELACIONES DE IMPEDANCIA}
\label{\detokenize{adaptacion/adaptacion:CIRCUITOS-DE-TRANSFORMADORES-Y-RELACIONES-DE-IMPEDANCIA}}
En general, es necesario controlar las impedancias de terminación de las lineas de señal de RF, especialmente en aplicaciones de banda ancha donde las longitudes de las lineas no son despresiables en relación con la longitud de onda. Los transformadores de RF de banda ancha se enrollan utilizando cables trenzados que se comportan como líneas de transmisión, y el acoplamiento requerido se produce a lo largo de estas líneas, así como magnéticamente a través del núcleo. El rendimiento óptimo se
logra cuando los devanados primario y secundario están conectados a impedancias de terminación resistivas para las cuales está diseñado el transformador. Los transformadores que tienen una relación de espiras de \(1:1\), por ejemplo, generalmente están diseñados para usarse en un sistema de \(50\) o \(75 \Omega\).

\sphinxincludegraphics[width=733\sphinxpxdimen,height=444\sphinxpxdimen]{{trafo_auto}.png}

En la figura, se ilustran tres topologías de devanado de transformador. El de la Figura 1a es el más simple. Este diseño, denominado autotransformador, tiene un devanado continuo roscado y no tiene aislamiento de CC. El transformador en la Figura 1b tiene bobinados primarios y secundarios separados, y proporciona aislamiento de CC. Sin embargo, el rendimiento de RF de estas configuraciones es similar.


\paragraph{Autotransformador}
\label{\detokenize{adaptacion/adaptacion:Autotransformador}}
Las ecuaciónes del autotransformador, segun la figura:
\begin{equation*}
\begin{split}\frac{v_o}{v_i} = \frac{i_i}{i_o} = \frac{N_a + N_b}{N_a}\end{split}
\end{equation*}
Las impedancias de entrada y de salida:
\begin{equation*}
\begin{split}\frac{Z_o}{Z_i} = \frac{\frac{v_o}{i_o}}{\frac{v_i}{i_i}} =  \frac{(N_a + N_b)^2}{N_a^2}\end{split}
\end{equation*}
En base a esta última ecuación, se puede obtener los valores de los inductores que conforman el transformador.
\begin{equation*}
\begin{split}\frac{L_o}{L_i} =  \frac{(N_a + N_b)^2}{N_a^2}\end{split}
\end{equation*}

\paragraph{Transformador}
\label{\detokenize{adaptacion/adaptacion:transformador-1}}\label{\detokenize{adaptacion/adaptacion:id1}}
Las ecuaciónes del transformador, segun la figura:
\begin{equation*}
\begin{split}\frac{v_o}{v_i} = \frac{i_i}{i_o} = \frac{N_b}{N_a}\end{split}
\end{equation*}
Las impedancias de entrada y de salida:
\begin{equation*}
\begin{split}\frac{Z_o}{Z_i} = \frac{\frac{v_o}{i_o}}{\frac{v_i}{i_i}} =  \frac{N_b^2}{N_a^2}\end{split}
\end{equation*}
En base a esta última ecuación, se puede obtener los valores de los inductores que conforman el transformador.
\begin{equation*}
\begin{split}\frac{L_o}{L_i} =  \frac{N_b^2}{N_a^2}\end{split}
\end{equation*}

\subsubsection{Ejemplo transformador}
\label{\detokenize{adaptacion/adaptacion:Ejemplo-transformador}}
\sphinxincludegraphics[width=482\sphinxpxdimen,height=166\sphinxpxdimen]{{trafo}.png}

Supongamos que es necesitamos un transformador para un circuito sintonizado en \(10 MHz\) con \(Q_c = 10\), empleando el mismo circuito. La resistencia de carga es de \(R_L = 10 \Omega\) y la del generador es de \(r_g =100 \Omega\).

El transformador en este caso debe presentar en sus bornes del bobinado primario \(r_g = 100 \Omega\) a \(r_g' =10 \Omega\) en el secundario.
\begin{equation*}
\begin{split}\frac{N_1}{N_2} = N = \sqrt{\frac{R_L^{'}}{R_L}} = 3.16\end{split}
\end{equation*}
donde \(N_1\) es la cantidad de espiras del primario, \(N_2\) es la cantidad de espiras del secundario, \(R_L^{'}\) es la resistencia de carga vista desde los bornes del primario (\(100 \Omega\)) y \(R_L\) (\(10 \Omega\)).

La resistencia total que carga al circuito sintonizado LC es de \(50 \Omega\) (formada por los \(100 \Omega\) de la fuente en paralelo con los \(100 \Omega\) que presenta el transformador). Dado que no se tienen en cuenta las perdidas, el \(Q_o = \inf\).

Entonces, la ecuación que nos permite calcular, el cual corresponde a el inductor en el primario:
\begin{equation*}
\begin{split}\frac{1}{Q_c} = \frac{1}{Q_o} + \frac{w_o \cdot L_p}{R_{ext}}\end{split}
\end{equation*}\begin{equation*}
\begin{split}\frac{1}{L_p} =   \frac{w_o \cdot Q_c}{R_{ext}\end{split}
\end{equation*}\begin{equation*}
\begin{split}R_{ext} = 50 \Omega\end{split}
\end{equation*}\begin{equation*}
\begin{split}L_p =  \frac{50 \Omega}{ 10 \cdot (2 \cdot \pi \cdot 10 MHz)} = 79.58 nHy\end{split}
\end{equation*}
Hasta aca conocemos la relación de espiras del transformador y el valor del indutor de este transformador en el secundario.

Si necesitamos simular este dispositivo empleando Spice, necesitamos conocer el valor de inductancia del secundario. Para esto podemos emplear la relación de espiras (esta ecuación en valida para K=1).
\begin{equation*}
\begin{split}L_s =  \frac{L_p}{N^2} = 7.958 nHy\end{split}
\end{equation*}
Para el calcular el valor de capacidad del capacitor.
\begin{equation*}
\begin{split}w_o^2 = \frac{1}{L\cdot C}\end{split}
\end{equation*}\begin{equation*}
\begin{split}C = \frac{1}{L\cdot (w_o^2 )} = 3.183 nF\end{split}
\end{equation*}

\paragraph{Simulando con LTSpice}
\label{\detokenize{adaptacion/adaptacion:Simulando-con-LTSpice}}
A continuación la simulación del circuito calculado y la respuesta.

Se midió la tensión sobre el primario, se buscó el ancho de banda para \(-3 dB\) y se obtuvo como resultado \(BW = 1 MHz\).

Por lo tanto,
\begin{equation*}
\begin{split}Q_c = \frac{f_o}{BW} = \frac{10 MHz}{1 MHz} = 10\end{split}
\end{equation*}
\sphinxincludegraphics[width=911\sphinxpxdimen,height=421\sphinxpxdimen]{{trafospice}.png}

\sphinxincludegraphics[width=1272\sphinxpxdimen,height=427\sphinxpxdimen]{{trafosimu}.png}

\sphinxurl{https://www.coilmaster.com.tw/comm/upfile/p\_160818\_07196.pdf}


\section{Divisor capacitivo}
\label{\detokenize{adaptacion/adaptacion:Divisor-capacitivo}}
Dado el circuito de la figura, realizaremos el analisis mediante conversiones serie\sphinxhyphen{}paralelo.

\sphinxincludegraphics[width=823\sphinxpxdimen,height=274\sphinxpxdimen]{{divC1a}.png}

Buscamos que el circuito presente una capacidad \(C\), \(R\) dada una \(R_o\).

Para el valor de \(C_2\).Para ello realizamos la conversión paralelo a serie, con lo que obtenemos el circuito de la figura.

\sphinxincludegraphics[width=957\sphinxpxdimen,height=264\sphinxpxdimen]{{divC2a}.png}

Para calcular los valores de \(R_os\) con \(Cs\), calculamos \(Q_{m2}\), partiendo de los valores de \(C\), \(R\) que son los que buscamos que presente el circuito (son datos).

Del circuito \(R\) y \(C\) paralelo:
\begin{equation*}
\begin{split}Q_{m2} = R \omega C\end{split}
\end{equation*}
La conversión de paralelo a serie:
\begin{equation*}
\begin{split}R_os = \frac{R}{(1+Q_{m2}^2)}\end{split}
\end{equation*}\begin{equation*}
\begin{split}Cs   =  C (1+\frac{1}{Q_{m2}^2})\end{split}
\end{equation*}
A partir del valor de \(R_os\) podemos calcular \(Q_{m1}\) (de ‘matching’) para llegar al paralelo de \(R_o\) y \(C_2\):
\begin{equation*}
\begin{split}R_o = R_os(1+Q_{m1}^2)\end{split}
\end{equation*}
Despejando el valor de \(Q_{m1}\)
\begin{equation*}
\begin{split}Q_{m1} =  \sqrt{\frac{R_o}{R_os}-1}\end{split}
\end{equation*}
Remplazando el valor de \(R_os\):
\begin{equation*}
\begin{split}Q_{m1} =  \sqrt{\frac{R_o}{R}(1+Q_{m2}^2)-1}\end{split}
\end{equation*}
A partir del valor de \(Q_{m1}\), calculamo \(C_2\)
\begin{equation*}
\begin{split}Q_{m1}  = R_o \omega C_2\end{split}
\end{equation*}\begin{equation*}
\begin{split}C_2 =  \frac{Q_{m1}}{R_o \omega}\end{split}
\end{equation*}
Entonces, planteadno la conversión de paralelo a serie.
\begin{equation*}
\begin{split}C_2s = C_2(1+\frac{1}{Q_{m1}^2})\end{split}
\end{equation*}
La serie de \(C_2s\) y \(C_1\) deben ser igual a \(C_s\)
\begin{equation*}
\begin{split}Cs = \frac{C_1C_2s}{C_1+C_2s}\end{split}
\end{equation*}
Despejando \(C_1\):
\begin{equation*}
\begin{split}C_1 = \frac{Cs C_2s}{Cs - C_2s}\end{split}
\end{equation*}
\sphinxstylestrong{Por lo tanto:}

Los datos son \(C\), \(R\) y \(R_o\).

Buscamos los valores de \(C_1\) y \(C_2\).
\begin{equation*}
\begin{split}Q_{m2} = R \omega C\end{split}
\end{equation*}\begin{equation*}
\begin{split}Q_{m1} = \sqrt[]{\frac{R_o}{R}(1+Q_{m2}^2)-1}\end{split}
\end{equation*}\begin{equation*}
\begin{split}C_2 = \frac{Q_{m1}}{R_o \omega}\end{split}
\end{equation*}\begin{equation*}
\begin{split}C_2s = C_2(1+\frac{1}{Q_{m1}^2})\end{split}
\end{equation*}\begin{equation*}
\begin{split}C   =  \frac{C}{ (1+\frac{1}{Q_{m2}^2})}\end{split}
\end{equation*}\begin{equation*}
\begin{split}C_1 = \frac{Cs C_2s}{C_2s - Cs}\end{split}
\end{equation*}

\subsection{Divisor capacitivo como autotransformador}
\label{\detokenize{adaptacion/adaptacion:Divisor-capacitivo-como-autotransformador}}
A partir de :\(Q_{m2} > 10\) y \(Q_{m1} > 10\).
\begin{equation*}
\begin{split}Q_{m1} = \sqrt[]{\frac{R_o}{R}(1+Q_{m2}^2)-1}\end{split}
\end{equation*}
Podemos llamar \(N^2 =\frac{R}{R_o}\), donde \(N\) será mayor a 1 ya que \(R > R_o\).
\begin{equation*}
\begin{split}Q_{m1} = \sqrt[]{\frac{(1+Q_{m2}^2)}{N^2}-1}\end{split}
\end{equation*}
Si ahora \(Q_{m2} > 10\), entonces:
\begin{equation*}
\begin{split}Q_{m1} = \sqrt[]{\frac{(Q_{m2}^2)}{N^2}-1}\end{split}
\end{equation*}
Donde si \(Q_{m1} > 10\), podemos escribir:
\begin{equation*}
\begin{split}Q_{m1} \sim \frac{Q_{m2}}{N}\end{split}
\end{equation*}
Calculo de \(C_2\)
\begin{equation*}
\begin{split}C_2 = \frac{Q_{m1}}{R_o \omega}\end{split}
\end{equation*}
Siendo \(Q_{m1}\):
\begin{equation*}
\begin{split}Q_{m1} \sim \frac{Q_{m2}}{N}\end{split}
\end{equation*}\begin{equation*}
\begin{split}C_2 \sim \frac{Q_{m2}}{N R_o \omega}\end{split}
\end{equation*}\begin{equation*}
\begin{split}C_2 \sim \frac{R \omega C}{N R_L \omega}\end{split}
\end{equation*}\begin{equation*}
\begin{split}C_2 \sim \frac{N^2 \omega C}{N \omega}\end{split}
\end{equation*}\begin{equation*}
\begin{split}C_2 \sim N C\end{split}
\end{equation*}
Calculo de \(C_1\)
\begin{equation*}
\begin{split}C_1 = \frac{C C_2s}{C - C_2s}\end{split}
\end{equation*}\begin{equation*}
\begin{split}C_2s \sim C_2\end{split}
\end{equation*}\begin{equation*}
\begin{split}C_1 = \frac{C C_2}{C - C_2}\end{split}
\end{equation*}\begin{equation*}
\begin{split}C_1 = \frac{N C}{N  - 1}\end{split}
\end{equation*}

\subsection{Procedimiento de calculo}
\label{\detokenize{adaptacion/adaptacion:Procedimiento-de-calculo}}\begin{equation*}
\begin{split}N =\sqrt[]{\frac{R}{R_o}}\end{split}
\end{equation*}\begin{equation*}
\begin{split}Q_{m2} = R \omega C\end{split}
\end{equation*}
Si \(Q_{m2} > 10\)
\begin{equation*}
\begin{split}Q_{m1} \sim \frac{Q_{m2}}{N}\end{split}
\end{equation*}
Si \(Q_{m1} > 10\)
\begin{equation*}
\begin{split}C_2 \sim N C\end{split}
\end{equation*}\begin{equation*}
\begin{split}C_1 = \frac{N C}{N  - 1}\end{split}
\end{equation*}
Si \(Q_{m1} \le 10\)

Volvemos a calcular \(Q_{m1}\):
\begin{equation*}
\begin{split}Q_{m1} = \sqrt[]{\frac{(1+Q_{m2}^2)}{N^2}-1}\end{split}
\end{equation*}
Teniendo el valor de \(Q_{m1}\):
\begin{equation*}
\begin{split}C_2  = \frac{Q_{m1}}{R_o \omega}\end{split}
\end{equation*}\begin{equation*}
\begin{split}C_2s = C_2(1+\frac{1}{Q_{m1}^2})\end{split}
\end{equation*}\begin{equation*}
\begin{split}Cs     = \frac{C}{ (1+\frac{1}{Q_{m2}^2})}\end{split}
\end{equation*}\begin{equation*}
\begin{split}C_1 = \frac{Cs C_2s}{C_2s - Cs}\end{split}
\end{equation*}

\subsubsection{Ejemplo divisor capacitivo}
\label{\detokenize{adaptacion/adaptacion:Ejemplo-divisor-capacitivo}}
En este ejemplo trabajamos con \(Q_{m1} > 10\) y \(Q_{m2} \le 10\).

Suponer que \(R = 8100 \Omega\), \(R_o = 100 \Omega\), \(f_o = 1.5 MHz\) y \(B = 100 KHz\). Suponer que el inductor tiene un factor de merito de \(Q_o = 40\). El generador tiene un resistencia de generador de \(r_g = 8100 \Omega\).

Se busca un ancho de banda de \(B = 100 KHz\) a una frecuencia de \(f_o = 1.5 MHz\). Diseñar para máxima transferencia de energía a \(Q\) constante.

Para un circuito RLC paralelo, podiamos calcular el \(Q_c\) del circuito como:
\begin{equation*}
\begin{split}Q_c = \frac{f_o}{B}\end{split}
\end{equation*}\begin{equation*}
\begin{split}Q_c= 15.00\end{split}
\end{equation*}
Entonces, para el caluclo de \(L\):
\begin{equation*}
\begin{split}w_o = 2 \pi f_o\end{split}
\end{equation*}\begin{equation*}
\begin{split}r_{ext} = \frac{rg R}{rg+R}\end{split}
\end{equation*}\begin{equation*}
\begin{split}XL = r_{ext} (\frac{1}{Q_c}-\frac{1}{Q_o})\end{split}
\end{equation*}\begin{equation*}
\begin{split}L  = \frac{XL}{w_o}\end{split}
\end{equation*}\begin{equation*}
\begin{split}L= 179\times 10^{-9} Hy\end{split}
\end{equation*}\begin{equation*}
\begin{split}XC = XL\end{split}
\end{equation*}\begin{equation*}
\begin{split}C  = \frac{1}{w_o XC}\end{split}
\end{equation*}\begin{equation*}
\begin{split}C= 629\times 10^{-12} C\end{split}
\end{equation*}
\sphinxstylestrong{Diseño del divisor capacitivo}
\begin{equation*}
\begin{split}N =\sqrt[]{\frac{R}{R_o}}\end{split}
\end{equation*}\begin{equation*}
\begin{split}N = 9.00\end{split}
\end{equation*}\begin{equation*}
\begin{split}Q_{m2} = R \omega C\end{split}
\end{equation*}\begin{equation*}
\begin{split}Q_m2 = 48.00\end{split}
\end{equation*}
\sphinxstylestrong{:math:\textasciigrave{}Q\_\{m2\} \textgreater{} 10\textasciigrave{}}
\begin{equation*}
\begin{split}Q_{m1} \sim \frac{Q_{m2}}{N}\end{split}
\end{equation*}\begin{equation*}
\begin{split}Q_m1 = 5.33\end{split}
\end{equation*}
\sphinxstylestrong{:math:\textasciigrave{}Q\_\{m1\} le 10\textasciigrave{}}

Volvemos a calcular \(Q_{m1}\):
\begin{equation*}
\begin{split}Q_{m1} = \sqrt[]{\frac{(1+Q_{m2}^2)}{N^2}-1}\end{split}
\end{equation*}\begin{equation*}
\begin{split}Q_m1 = 5.24\end{split}
\end{equation*}
Teniendo el valor de \(Q_{m1}\):
\begin{equation*}
\begin{split}C_2  = \frac{Q_{m1}}{R_o \omega}\end{split}
\end{equation*}\begin{equation*}
\begin{split}C_2= 5.56\times 10^{-9} F\end{split}
\end{equation*}\begin{equation*}
\begin{split}C_2s = C_2(1+\frac{1}{Q_{m1}^2})\end{split}
\end{equation*}\begin{equation*}
\begin{split}C_2s= 5.76\times 10^{-9} F\end{split}
\end{equation*}\begin{equation*}
\begin{split}Cs     = \frac{C}{ (1+\frac{1}{Q_{m2}^2})}\end{split}
\end{equation*}\begin{equation*}
\begin{split}C_s= 628 \times 10^{-12} F\end{split}
\end{equation*}\begin{equation*}
\begin{split}C_1 = \frac{Cs C_2s}{C_2s - Cs}\end{split}
\end{equation*}\begin{equation*}
\begin{split}C1= 705\times 10^{-12} F\end{split}
\end{equation*}
\sphinxincludegraphics[width=432\sphinxpxdimen,height=288\sphinxpxdimen]{{ejemplodivCa}.png}


\subsubsection{Ejemplo divisor capacitivo \protect\(Q_{m1} > 10\protect\) y \protect\(Q_{m2} > 10\protect\)}
\label{\detokenize{adaptacion/adaptacion:Ejemplo-divisor-capacitivo-Q__m1_->-10-y-Q__m2_->-10}}
Suponer \(r_g = 10 K\Omega\), \(R_o = 1 K\Omega\), \(f_o = 10.7 MHz\) y \(B= 200KHz\). El inductor tiene un factor de selectividad de \(Q_o = 80\).

Para un circuito RLC paralelo, podiamos calcular el \(Q_c\) del circuito como:
\begin{equation*}
\begin{split}Q_c = \frac{f_o}{B}\end{split}
\end{equation*}\begin{equation*}
\begin{split}Q_c= 53.50\end{split}
\end{equation*}
Entonces, para el caluclo de \(L\):
\begin{equation*}
\begin{split}w_o = 2 \pi f_o\end{split}
\end{equation*}\begin{equation*}
\begin{split}r_{ext} = \frac{rg R}{rg+R}\end{split}
\end{equation*}\begin{equation*}
\begin{split}XL = r_{ext} (\frac{1}{Q_c}-\frac{1}{Q_o})\end{split}
\end{equation*}\begin{equation*}
\begin{split}L  = \frac{XL}{w_o}\end{split}
\end{equation*}\begin{equation*}
\begin{split}L= 460\times 10^{-9} Hy\end{split}
\end{equation*}\begin{equation*}
\begin{split}XC = XL\end{split}
\end{equation*}\begin{equation*}
\begin{split}C  = \frac{1}{w_o XC}\end{split}
\end{equation*}\begin{equation*}
\begin{split}C= 480\times 10^{-12} F\end{split}
\end{equation*}
\sphinxstylestrong{Diseño del divisor capacitivo}
\begin{equation*}
\begin{split}N =\sqrt[]{\frac{R}{R_o}}\end{split}
\end{equation*}\begin{equation*}
\begin{split}N = 3.16\end{split}
\end{equation*}\begin{equation*}
\begin{split}Q_{m2} = R \omega C\end{split}
\end{equation*}\begin{equation*}
\begin{split}Q_m2 = 323.02\end{split}
\end{equation*}
Si \(Q_{m2} > 10\)
\begin{equation*}
\begin{split}Q_{m1} \sim \frac{Q_{m2}}{N}\end{split}
\end{equation*}\begin{equation*}
\begin{split}Q_m1 = 102.15\end{split}
\end{equation*}
Si \(Q_{m1} > 10\)
\begin{equation*}
\begin{split}C_2 \sim N C\end{split}
\end{equation*}\begin{equation*}
\begin{split}C_2= 1.52\times 10^{-9} F\end{split}
\end{equation*}\begin{equation*}
\begin{split}C_1 = \frac{N C}{N  - 1}\end{split}
\end{equation*}\begin{equation*}
\begin{split}C_1= 703\times 10^{-12} F\end{split}
\end{equation*}
\sphinxincludegraphics[width=432\sphinxpxdimen,height=288\sphinxpxdimen]{{ejemplodivCb}.png}


\section{Filtro PI}
\label{\detokenize{adaptacion/adaptacion:Filtro-PI}}
Dado el circuito de la figura, realizaremos el analisis mediante conversiones serie\sphinxhyphen{}paralelo.

\sphinxincludegraphics[width=568\sphinxpxdimen,height=227\sphinxpxdimen]{{ejemploPIa}.png}

Buscamos que el circuito presente una resistencia \(R\) dada una \(R_o\) a la frecuancia de sintonia, con un determinado \(Q_c\).

\sphinxincludegraphics[width=454\sphinxpxdimen,height=199\sphinxpxdimen]{{ejemploPIe}.png}

Empezando por este último y suponiendo que el inductor tiene un factor de merito de \(Q_o\).
\begin{equation*}
\begin{split}\frac{1}{Q_c} =  \frac{1}{Q_o} + \frac{w L}{R_{ext}}\end{split}
\end{equation*}
donde \(R_{ext}\) corresponde a las resistencias totales que cierran el circuito con masa (\(R\) y \(r_g\) por ejemplo).

Entonces, \(C_1\) sintoniza con \(L\).

\sphinxincludegraphics[width=442\sphinxpxdimen,height=217\sphinxpxdimen]{{ejemploPId}.png}

Dado \(L\), podemos calcular el \(Q_{m2}\), para la conversión paralelo a serie de \(R\) y \(L\).
\begin{equation*}
\begin{split}Q_{m2} = \frac{R}{\omega L}\end{split}
\end{equation*}
Obteniendo de esta manera \(L_1a\) y \(R_os\).
\begin{equation*}
\begin{split}L_1a = \frac{L}{(1+\frac{1}{Q_{m2}^2})}\end{split}
\end{equation*}\begin{equation*}
\begin{split}R_os = \frac{R}{(1+Q_{m2}^2)}\end{split}
\end{equation*}
\sphinxincludegraphics[width=581\sphinxpxdimen,height=232\sphinxpxdimen]{{ejemploPIb}.png}

De igual manera, desde la salida

Del circuito \(R_o\) y \(C_2\) paralelo:
\begin{equation*}
\begin{split}Q_{m1} = R_o \omega C_2\end{split}
\end{equation*}
La conversión de paralelo a serie, que debe coincidir con el valor de conversión encontrado \(R_os\).
\begin{equation*}
\begin{split}R_os = \frac{R_o}{(1+Q_{m1}^2)}\end{split}
\end{equation*}
Despejando \(Q_{m1}\)
\begin{equation*}
\begin{split}Q_{m1} = \sqrt{\frac{R_o}{R_os}-1}\end{split}
\end{equation*}\begin{equation*}
\begin{split}C_2s   =  C_2 (1+\frac{1}{Q_{m1}^2})\end{split}
\end{equation*}
\sphinxincludegraphics[width=562\sphinxpxdimen,height=231\sphinxpxdimen]{{ejemploPIc}.png}

Solo queda neutralizar el capacitor \(C_2s\) con un indictor que llamamo \(L_1b\).

Por último, el valor de \(L_1\) es la suma de ambos inductores.


\subsection{Ejemplo filtro PI}
\label{\detokenize{adaptacion/adaptacion:Ejemplo-filtro-PI}}
\sphinxincludegraphics[width=589\sphinxpxdimen,height=250\sphinxpxdimen]{{ejemploPIeje1}.png}

Suponer que \(R = 100 \Omega\), \(R_o = 100 \Omega\) y \(f_o = 100 MHz\). Suponer que el inductor tiene un factor de merito de \(Q_o = 40\). El generador tiene un resistencia de generador de \(r_g = 100 \Omega\).

Diseñar para máxima transferencia de energía a \(Q\) constante.

Empezando por este último y suponiendo que el inductor tiene un factor de merito de \(Q_o\).
\begin{equation*}
\begin{split}\frac{1}{Q_c} =  \frac{1}{Q_o} + \frac{w L}{R_{ext}}\end{split}
\end{equation*}\begin{equation*}
\begin{split}R_{ext} = \frac{rg R}{(rg+R)}\end{split}
\end{equation*}\begin{equation*}
\begin{split}XL = rext (\frac{1}{Q_c}-\frac{1}{Q_o})\end{split}
\end{equation*}\begin{equation*}
\begin{split}L  = \frac{XL}{w_o}\end{split}
\end{equation*}\begin{equation*}
\begin{split}L= 5.97 \times 10^{-9} Hy\end{split}
\end{equation*}\begin{equation*}
\begin{split}XC1 = XL\end{split}
\end{equation*}\begin{equation*}
\begin{split}C1  = \frac{1}{(wo XC1)}\end{split}
\end{equation*}\begin{equation*}
\begin{split}C1= 424 \times 10^{-12} F\end{split}
\end{equation*}\begin{equation*}
\begin{split}Q_{m2} = \frac{R}{(wo  L)}\end{split}
\end{equation*}\begin{equation*}
\begin{split}Q_m2 = 26.67\end{split}
\end{equation*}\begin{equation*}
\begin{split}L_1a = \frac{L}{(1+\frac{1}{Q_{m2}^2})}\end{split}
\end{equation*}\begin{equation*}
\begin{split}L_1a = 5.96 \times 10^{-9} Hy\end{split}
\end{equation*}\begin{equation*}
\begin{split}R_os = \frac{R}{(1+  Q_{m2}^2)}\end{split}
\end{equation*}\begin{equation*}
\begin{split}R_os = 0.14 \Omega\end{split}
\end{equation*}\begin{equation*}
\begin{split}Q_{m1} = \sqrt{(\frac{R_o}{R_os})-1}\end{split}
\end{equation*}\begin{equation*}
\begin{split}Q_m1 = 18.84\end{split}
\end{equation*}\begin{equation*}
\begin{split}C_2 = \frac{Q_{m1}}{Ro wo}\end{split}
\end{equation*}\begin{equation*}
\begin{split}C2= 600 \times 10^{-12} F\end{split}
\end{equation*}\begin{equation*}
\begin{split}C_2s = C_2 (1+\frac{1}{Q_{m1}^2})\end{split}
\end{equation*}\begin{equation*}
\begin{split}L_1b = \frac{1}{(C2s wo^2)}\end{split}
\end{equation*}\begin{equation*}
\begin{split}L_1b = 4.21\times 10^{-9} Hy\end{split}
\end{equation*}\begin{equation*}
\begin{split}L_1 = L_1a + L_1b\end{split}
\end{equation*}\begin{equation*}
\begin{split}L_1 = 10.2\times 10^{-9} Hy\end{split}
\end{equation*}
\sphinxincludegraphics[width=640\sphinxpxdimen,height=480\sphinxpxdimen]{{ejemploPI}.png}

{
\sphinxsetup{VerbatimColor={named}{nbsphinx-code-bg}}
\sphinxsetup{VerbatimBorderColor={named}{nbsphinx-code-border}}
\begin{sphinxVerbatim}[commandchars=\\\{\}]
\llap{\color{nbsphinxin}[ ]:\,\hspace{\fboxrule}\hspace{\fboxsep}}
\end{sphinxVerbatim}
}


\chapter{Amplificador generico}
\label{\detokenize{sintonizados/sintonizados:Amplificador-generico}}\label{\detokenize{sintonizados/sintonizados::doc}}
El diseño del amplificador sintonizado de pequeña señal de RF generalmente se basa en el requisito de una ganancia de potencia específica a una frecuencia dada. Otros objetivos de diseño pueden incluir ancho de banda, estabilidad, aislamiento de entrada\sphinxhyphen{}salida y bajo rendimiento de ruido. Después de seleccionar un tipo de circuito básico, se pueden resolver las ecuaciones de diseño aplicables. Los circuitos se pueden clasificar de acuerdo con la retroalimentación (neutralización,
unilateralización o sin retroalimentación), y la coincidencia en los terminales del transistor (las admisiones del circuito coinciden o no con las entradas y salidas de los transistores). Se discutirá cada una de estas categorías de circuitos, incluidas las ecuaciones de diseño aplicables y las consideraciones que conducen a la selección de una configuración particular.


\section{Consideranciones generales de diseño}
\label{\detokenize{sintonizados/sintonizados:Consideranciones-generales-de-dise_xf1o}}
Las ecuaciones que figuran en el texto de este informe son aplicables a las configuraciones de emisor común, base común o colector común, utilizando el conjunto de parámetros correspondiente (parámetros de emisor común, base común o colector común). Si bien se desarrolla principalmente el diseño de circuitos con transistores bipolares convencionales, la teoría de la red de dos puertos tiene la ventaja de ser aplicable a cualquier red activa lineal. Por lo tanto, el mismo enfoque de diseño y
ecuaciones pueden usarse con los transistores de efecto de campo, los circuitos integrados o cualquier otro dispositivo que pueda describirse como una red activa lineal de dos puertos.

Considere un amplificador genérico de dos puertos que se muestra en la figura. Los circuitos de dos puertos lineales e invariante en el tiempo se puede describir usando cualquier conjunto de parámetros de dos puertos, incluidos los parámetros de admitancia Y, parámetros de impedancia Z, parámetros híbridos H o los parámetros de dispersión S.

\sphinxincludegraphics[width=643\sphinxpxdimen,height=327\sphinxpxdimen]{{sistema}.png}

Los parámetros son genéricos e independientes de los detalles del amplificador, puede ser un solo transistor o un amplificador de etapas múltiples. Ademas, los transistores de alta frecuencia se describen más fácilmente mediante parámetros de dos puertos Los amplificadores realimentación a menudo se pueden descomponer en un amplificador unilateral equivalente de dos puertos y una sección de retroalimentación de dos puertos. Podemos sacar algunas conclusiones muy generales sobre la ganancia de
potencia “óptima” de un puerto de dos puertos, lo que nos permite definir algunas métricas útiles.


\section{Parametro admitancia}
\label{\detokenize{sintonizados/sintonizados:Parametro-admitancia}}
El circuito de un cuadripolo admitancia se muestra en la figura.

\sphinxincludegraphics[width=668\sphinxpxdimen,height=198\sphinxpxdimen]{{admitancia1}.png}

Las ecuaciones del cuadripolo en función de los parametros de admitancia y tensiones del circuito:
\begin{equation*}
\begin{split}i_i = v_i\cdot y_{11} + v_o \cdot y_{12}\end{split}
\end{equation*}\begin{equation*}
\begin{split}i_o = v_i\cdot y_{21} + v_o \cdot y_{22}\end{split}
\end{equation*}

\subsection{Admintacia de entrada}
\label{\detokenize{sintonizados/sintonizados:Admintacia-de-entrada}}
\sphinxincludegraphics[width=803\sphinxpxdimen,height=209\sphinxpxdimen]{{admiEntrada1}.png}

Del circuito, se puede calcular la admitacia de entrada dada una admitancia de salida \(y_{L}\):
\begin{equation*}
\begin{split}y_{in}=y_{11}- \frac {y_{12}y_{21}}{y_{22}+y_{L}}\end{split}
\end{equation*}

\subsection{Admintacia de salida}
\label{\detokenize{sintonizados/sintonizados:Admintacia-de-salida}}
\sphinxincludegraphics[width=750\sphinxpxdimen,height=212\sphinxpxdimen]{{admiSalida1}.png}

Del circuito, se puede calcular la admitacia de entrada dada una admitancia de salida \(y_{L}\):
\begin{equation*}
\begin{split}y_{out}=y_{22}- \frac {y_{12}y_{21}}{y_{11}+y_{g}}\end{split}
\end{equation*}

\subsection{Modelos equivalente para alta frecuencia}
\label{\detokenize{sintonizados/sintonizados:Modelos-equivalente-para-alta-frecuencia}}
El modelo híbrido\sphinxhyphen{}pi puede ser bastante exacto para los circuitos de baja frecuencia y puede ser adaptado para circuitos de frecuencia más alta con el agregado de capacitancias y otros elementos parásitos al modelo.

\sphinxincludegraphics[width=724\sphinxpxdimen,height=251\sphinxpxdimen]{{modeloH}.png}

El modelo híbrido\sphinxhyphen{}pi puede relacionarse con los parámetros admitancia.

\sphinxincludegraphics[width=449\sphinxpxdimen,height=180\sphinxpxdimen]{{modeloH_Y}.png}


\chapter{Técnicas de análisis de circuitos simple sintonizado.}
\label{\detokenize{sintonizados/sintonizados:T_xe9cnicas-de-an_xe1lisis-de-circuitos-simple-sintonizado.}}
Los circuitos sintonizados son una clase importante de circuitos que se encuentran en todos los transceptores inalámbricos. Los amplificadores sintonizados se emplean para amplificar un rango de frecuencias. Estos amplificadores al emplear circuitos sintonizados presentan la respuesta en frecuencia de un filtro pasabanda. Como ventaja, los circuitos \(LC\) permiten compensar algunas de los componentes parásitos de los dispositivos activos.

Como veremos, se pueden describir como filtros de paso y de segundo orden cuyo rendimiento se puede analizar y diseñar de manera muy similar a la de los amplificadores de baja frecuencia.

\sphinxincludegraphics[width=462\sphinxpxdimen,height=227\sphinxpxdimen]{{topologiaNPN}.png}

La figura muestra un amplificador realizado con un transistor NPN y etapas de sintonia. Cada una de estas etapas corresponde a un simple sintonizado.

A continuación, derivaremos expresiones analíticas para la ganancia de voltaje y la ganancia de potencia de las etapas de amplificador más comunes. Estas expresiones relativamente simples proporcionan información útil sobre el funcionamiento de los amplificadores sintonizados y un punto de partida bastante preciso para el diseño por computadora de amplificadores sintonizados.


\section{Ganancia de tensión de una etapa simple sintonizada}
\label{\detokenize{sintonizados/sintonizados:Ganancia-de-tensi_xf3n-de-una-etapa-simple-sintonizada}}
Una etapa simple sintonizado se conforma por filtro \(LC\), que también suele diseñarse para adaptar las impedancias de la etapa.

Consideremos la topología básica, como la que se muestra en la Figura, de una etapa CE o CS con un carga RLC paralela resonante (circuito de sintonización simple). Para simplificar el análisis, la resistencia de salida del transistor y la capacitancia se incorporan en \(R\) y \(C1\).

\sphinxincludegraphics[width=736\sphinxpxdimen,height=209\sphinxpxdimen]{{simple}.png}

Al igual que a bajas frecuencias, e ignorando la capacitancia de Miller, la ganancia de tensión para el circuito simple sintonizado se expresa como:
\begin{equation*}
\begin{split}v_{o} =  -g_m \cdot v_{i} \cdot \frac{1}{(\frac{1}{R} +\frac{1}{SL_1} + SC_1)}\end{split}
\end{equation*}\begin{equation*}
\begin{split}v_o =   \frac{-g_m \cdot v_i  }{(S\,C_1+\frac{1}{R}+\frac{1}{S\,L_1} )}\end{split}
\end{equation*}\begin{equation*}
\begin{split}\frac{v_o}{v_i}=     \frac{-g_m}{C_1} \cdot  \frac{S}{(S^2 +\frac{S}{R\cdot C_1}+\frac{1}{C_1\,L_1} )}\end{split}
\end{equation*}
Donde podemos normalizar la ecuación empleando los terminos \(Q\), ya presentado, y \(\omega_o^2 = \frac{1}{LC}\) como la frecuencia de resonancia.
\begin{equation*}
\begin{split}\frac{v_o}{v_i}=     \frac{-g_m}{C_1} \cdot  \frac{S}{(S^2 +\frac{S}{R\cdot C_1}+\omega_o^2 )}\end{split}
\end{equation*}
Podemos remplazar el \(C_1 = \frac{Q}{R \cdot \omega_o}\)
\begin{equation*}
\begin{split}\frac{v_o}{v_i}=     \frac{-g_m R \cdot \omega_o}{Q} \cdot  \frac{S}{(S^2 +\frac{S \omega_o}{Q}+\omega_o^2 )}\end{split}
\end{equation*}

\section{Modulo y fase de la transferencia de tensión}
\label{\detokenize{sintonizados/sintonizados:Modulo-y-fase-de-la-transferencia-de-tensi_xf3n}}
La trasferencia de un simple sintonizado se obtuvo como
\begin{equation*}
\begin{split}A_v = \frac{-g_m R \cdot \omega_o}{Q} \frac{S}{S^2 + \frac{S\omega_o}{Q} + \omega_o^2}\end{split}
\end{equation*}
Para conocer la respuesta en frecuencia de la trasferencia, debemos remplazar \(S = j \omega\), donde \(\omega\) es la variable.
\begin{equation*}
\begin{split}A_v = \frac{-g_m R \cdot \omega_o}{Q} \frac{j\omega}{(j\omega)^2 + \frac{j\omega \omega_o}{Q} + \omega_o^2}\end{split}
\end{equation*}
Operando.
\begin{equation*}
\begin{split}A_v = \frac{-g_m R \cdot \omega_o}{Q} \frac{j\omega}{ \frac{j\omega \omega_o}{Q} + \omega_o^2 -\omega^2}\end{split}
\end{equation*}\begin{equation*}
\begin{split}A_v =     \frac{-g_m R}{ 1 + jQ(\frac{\omega^2 - \omega_o^2}{\omega\omega_o}) }\end{split}
\end{equation*}
Se obtiene entonces la trasferencia
\begin{equation*}
\begin{split}A_v(\omega) =     \frac{-g_m R}{1 + jQ(\frac{\omega}{\omega_o}-\frac{ \omega_o}{\omega}) }\end{split}
\end{equation*}
El modulo de la trasferencia
\begin{equation*}
\begin{split}|A_v|(\omega) =   \frac{g_m R}{ \sqrt[2]{1 + Q^2(\frac{\omega}{\omega_o}-\frac{ \omega_o}{\omega})^2} }\end{split}
\end{equation*}
La fase de la trasnferecnia
\begin{equation*}
\begin{split}\Phi(\omega) = \pi - arctng(Q(\frac{\omega}{\omega_o}-\frac{ \omega_o}{\omega}))\end{split}
\end{equation*}

\subsection{Normalización de la trasferencia}
\label{\detokenize{sintonizados/sintonizados:Normalizaci_xf3n-de-la-trasferencia}}
Dado que para el diseño de los amplificadores sintonizados es necesario conocer el rechazo que tendran algunas frecuencias respecto a la frecuencia de sintonia, es útil para esto emplear la trasferencia normalizada. Esta se obtiene mediante la relación entre la trasferencia respecto a la trasferencia a la frecuencia de sintonia.
\begin{equation*}
\begin{split}|\bar{A_v}|(\omega) =  \frac{|A_v|(\omega)}{|A_v|(\omega_o)}\end{split}
\end{equation*}
donde \(|A_v|(\omega_o) = g_m R\).
\begin{equation*}
\begin{split}|\bar{A_v}|(\omega) =  \frac{\frac{g_m R}{ \sqrt[2]{1 + Q^2(\frac{\omega}{\omega_o}-\frac{ \omega_o}{\omega})^2} }}{ g_m R}\end{split}
\end{equation*}\begin{equation*}
\begin{split}|\bar{A_v}|(\omega) =   \frac{1}{ \sqrt[2]{1 + Q^2(\frac{\omega}{\omega_o}-\frac{ \omega_o}{\omega})^2} }\end{split}
\end{equation*}
Que también puede ser expresada en función de la frecuencia como
\begin{equation*}
\begin{split}|\bar{A_v}|(f) =   \frac{1}{ \sqrt[2]{1 + Q^2(\frac{f}{f_o}-\frac{ f_o}{f})^2} }\end{split}
\end{equation*}

\subsection{Simetría}
\label{\detokenize{sintonizados/sintonizados:Simetr_xeda}}\begin{equation*}
\begin{split}|A_v| =   \frac{g_m R}{ \sqrt[2]{1 + Q^2(\frac{\omega}{\omega_o}-\frac{ \omega_o}{\omega})^2} }\end{split}
\end{equation*}
Para una atenuación dada:
\begin{equation*}
\begin{split}\frac{g_m R}{ \sqrt[2]{1 + Q^2(\frac{\omega_i}{\omega_o}-\frac{ \omega_o}{\omega_i})^2} } = \frac{g_m R}{ \sqrt[2]{1 + Q^2(\frac{\omega_s}{\omega_o}-\frac{ \omega_o}{\omega_s})^2} }\end{split}
\end{equation*}\begin{equation*}
\begin{split}Q^2(\frac{\omega_i}{\omega_o}-\frac{ \omega_o}{\omega_i})^2  = Q^2(\frac{\omega_s}{\omega_o}-\frac{ \omega_s}{\omega_2})^2\end{split}
\end{equation*}\begin{equation*}
\begin{split}\omega_1 \omega_2  = \omega_o^2\end{split}
\end{equation*}
Simetría geométrica.


\section{Ejemplo 1}
\label{\detokenize{sintonizados/sintonizados:Ejemplo-1}}
Supongamos una etapa simple sintonizada a la frecuencia de \(f_o = 1MHz\), con un factor de selectividad de \(Q_c = 10\). El elemento activo tiene una ganancia de transconductancia \(g_m = 100 mS\) y la resistencia total de la etapa es de \(R_t = 100 \Omega\).

La transferencia de tensión en función de la frecuencia para este sistema (en escala semilog en la frecuecia).

\sphinxincludegraphics[width=432\sphinxpxdimen,height=288\sphinxpxdimen]{{ejemplo1AvABS}.png}

La transfencia en dB

\sphinxincludegraphics[width=432\sphinxpxdimen,height=288\sphinxpxdimen]{{ejemplo1AvdB}.png}


\section{Diagrama de polos y ceros de un simple sintonizado}
\label{\detokenize{sintonizados/sintonizados:Diagrama-de-polos-y-ceros-de-un-simple-sintonizado}}
Volviendo a la expresión
\begin{equation*}
\begin{split}A_v = - \frac{g_m}{C} \frac{S}{S^2 + \frac{S}{CR} + \frac{1}{C L}}\end{split}
\end{equation*}
La expresión puede ser rescrita como:
\begin{equation*}
\begin{split}A_v = - \frac{g_m}{C} \frac{S}{(S-p_1)(S-p_2)}\end{split}
\end{equation*}
Donde los polos:
\begin{equation*}
\begin{split}p_1 = -\frac{1}{2RC} + \sqrt[2]{\frac{1}{4R^2C^2} - \frac{1}{LC} }\end{split}
\end{equation*}\begin{equation*}
\begin{split}p_2 = -\frac{1}{2RC} - \sqrt[2]{\frac{1}{4R^2C^2} - \frac{1}{LC} }\end{split}
\end{equation*}
Donde podemos remplazar los siguiente terminos:
\begin{equation*}
\begin{split}\omega_o^2 = \frac{1}{LC}\end{split}
\end{equation*}\begin{equation*}
\begin{split}Q = \omega_o C R\end{split}
\end{equation*}\begin{equation*}
\begin{split}p_1 = -\frac{\omega_o }{2Q} + \sqrt[2]{\frac{\omega_o^2}{4Q^2} - \omega_o^2 }\end{split}
\end{equation*}\begin{equation*}
\begin{split}p_2 = -\frac{\omega_o }{2Q} - \sqrt[2]{\frac{\omega_o^2}{4Q^2} - \omega_o^2 }\end{split}
\end{equation*}
Factor comun \(-\omega_o^2\):
\begin{equation*}
\begin{split}p_1 = -\frac{\omega_o }{2Q} + j\omega_o \sqrt[2]{ 1 -\frac{1}{4Q^2}}\end{split}
\end{equation*}\begin{equation*}
\begin{split}p_2 = -\frac{\omega_o }{2Q} - j\omega_o \sqrt[2]{ 1 -\frac{1}{4Q^2}}\end{split}
\end{equation*}
Si \(Q > 2\), podemos aproximar los polos a:
\begin{equation*}
\begin{split}p_1 = -\frac{\omega_o }{2Q} + j\omega_o\end{split}
\end{equation*}\begin{equation*}
\begin{split}p_2 = -\frac{\omega_o }{2Q} - j\omega_o\end{split}
\end{equation*}
Siendo entonces la ganancia de tensión:
\begin{equation*}
\begin{split}A_v = - \frac{g_m}{C} \frac{S}{(S+\frac{\omega_o }{2Q} + j\omega_o)(S+\frac{\omega_o }{2Q} - j\omega_o )}\end{split}
\end{equation*}
\sphinxincludegraphics[width=144\sphinxpxdimen,height=360\sphinxpxdimen]{{ejemplo1PZ}.png}


\section{Ancho de banda}
\label{\detokenize{sintonizados/sintonizados:Ancho-de-banda}}
Las frecuencias donde la trasferencia
\begin{equation*}
\begin{split}\frac{g_m R}{ \sqrt[2]{1 + Q^2(\frac{\omega_c}{\omega_o}-\frac{ \omega_o}{\omega_c})^2} } = \frac{1}{\sqrt[2]{2}}\end{split}
\end{equation*}\begin{equation*}
\begin{split}1 + Q^2(\frac{\omega_c}{\omega_o}-\frac{ \omega_o}{\omega_c})^2  = 2\end{split}
\end{equation*}\begin{equation*}
\begin{split}Q^2(\frac{\omega_c}{\omega_o}-\frac{ \omega_o}{\omega_c})^2  = 1\end{split}
\end{equation*}\begin{equation*}
\begin{split}Q(\frac{\omega_c}{\omega_o}-\frac{ \omega_o}{\omega_c})  = \pm 1\end{split}
\end{equation*}\begin{equation*}
\begin{split}\omega_c^2-\omega_o^2   =\pm \frac{\omega_o\omega_c}{Q}\end{split}
\end{equation*}\begin{equation*}
\begin{split}\omega_c^2    \pm \omega_c \frac{\omega_o}{Q} + \omega_o^2 = 0\end{split}
\end{equation*}\begin{equation*}
\begin{split}\omega_c = \pm \frac{\omega_o}{2Q} \pm \sqrt[]{\frac{\omega_o^2}{4Q^2}- \omega_o^2}\end{split}
\end{equation*}\begin{equation*}
\begin{split}\omega_c = \pm \frac{\omega_o}{2Q} \pm \omega_o \sqrt[]{\frac{1}{4Q^2}- 1}\end{split}
\end{equation*}
Como tienen que ser frecuencias positivas:
\begin{equation*}
\begin{split}\omega_{c_{i,s}} = \pm \frac{\omega_o}{2Q} \pm \omega_o \sqrt[]{\frac{1}{4Q^2}- 1}\end{split}
\end{equation*}\begin{equation*}
\begin{split}\omega_{c_{i,s}} = \omega_o \sqrt[]{\frac{1}{4Q^2}- 1} \pm \frac{\omega_o}{2Q}\end{split}
\end{equation*}\begin{equation*}
\begin{split}\omega_{c_{i,s}} \simeq \omega_o  \pm \frac{\omega_o}{2Q}\end{split}
\end{equation*}\begin{equation*}
\begin{split}\omega_{c_{i}} \simeq \omega_o  - \frac{\omega_o}{2Q}\end{split}
\end{equation*}\begin{equation*}
\begin{split}\omega_{c_{s}} \simeq \omega_o  - \frac{\omega_o}{2Q}\end{split}
\end{equation*}\begin{equation*}
\begin{split}BW = \omega_{c_{s}} -\omega_{c_{i}} = \frac{\omega_o}{Q}\end{split}
\end{equation*}

\section{Aproximación de banda angosta}
\label{\detokenize{sintonizados/sintonizados:Aproximaci_xf3n-de-banda-angosta}}
Partiendo de la respuesta en frecuencia del sistema.
\begin{equation*}
\begin{split}A_v = - \frac{g_m}{C} \frac{S}{(S+\frac{\omega_o }{2Q} + j\omega_o)(S+\frac{\omega_o }{2Q} - j\omega_o )}\end{split}
\end{equation*}\begin{equation*}
\begin{split}A_v = - \frac{g_m}{C} \frac{S}{(S-p_1)(S-p_2)}\end{split}
\end{equation*}
\sphinxincludegraphics[width=203\sphinxpxdimen,height=525\sphinxpxdimen]{{simple_ABA}.png}

El diagrama muestra la respuesta en frecuencia del sistema para una frecuencia dada. Para esta frecuencia la transferencia puede ser calculada como:
\begin{equation*}
\begin{split}A_v(\omega_x) = - \frac{g_m}{C} \frac{S_o}{S_1 S_2}\end{split}
\end{equation*}
Para simplificar el analisis, supondremos que los vectores \(S_o\) y \(S_2\) tienen una variación despresible respecto a las variaciones de \(S_1\). Luego demostraremos los limites de esta suposición.

Entonces:
\begin{equation*}
\begin{split}S_o \sim  j\omega_o\end{split}
\end{equation*}\begin{equation*}
\begin{split}S_2 \sim  2j\omega_o\end{split}
\end{equation*}
donde el vector que varia es \(S_1\)
\begin{equation*}
\begin{split}S_1 = j\omega + \frac{\omega_o }{2Q} + j\omega_o\end{split}
\end{equation*}
Remplazando en la ecuación de la trasnferencia
\begin{equation*}
\begin{split}A_v = - \frac{g_m}{C} \frac{j\omega_o}{(j\omega+\frac{\omega_o }{2Q} + j\omega_o^2)(j2\omega_o)}\end{split}
\end{equation*}\begin{equation*}
\begin{split}A_v = - \frac{g_m}{C} \frac{1}{(j\omega+\frac{\omega_o }{2Q} + j\omega_o)(2)}\end{split}
\end{equation*}\begin{equation*}
\begin{split}A_v = - g_m R \frac{1}{1 + j2Q \frac{\omega-\omega_o }{\omega_o}}\end{split}
\end{equation*}\begin{equation*}
\begin{split}A_v = - g_m R \frac{1}{1 + j\chi}\end{split}
\end{equation*}\begin{equation*}
\begin{split}\chi(\omega) = 2Q \frac{\omega-\omega_o }{\omega_o}\end{split}
\end{equation*}\begin{equation*}
\begin{split}|A_v| =   \frac{g_m R}{\sqrt[]{1 + \chi^2}}\end{split}
\end{equation*}\begin{equation*}
\begin{split}|\overline{A_v}| =   \frac{1}{\sqrt[]{1 + \chi^2}}\end{split}
\end{equation*}

\section{Ejemplo 2}
\label{\detokenize{sintonizados/sintonizados:Ejemplo-2}}
Supongamos una etapa simple sintonizada a la frecuencia de \(f_o = 1MHz\), con un factor de selectividad de \(Q_c = 10\). El elemento activo tiene una ganancia de transconductancia \(g_m = 100 mS\) y la resistencia total de la etapa es de \(R_t = 100 \Omega\).

La transferencia de tensión en función de la frecuencia para este sistema empleando aproximación de banda angosta (en escala semilog en la frecuecia).

\sphinxincludegraphics[width=432\sphinxpxdimen,height=288\sphinxpxdimen]{{ejemplo2ABA}.png}

La transfencia en dB

\sphinxincludegraphics[width=432\sphinxpxdimen,height=288\sphinxpxdimen]{{ejemplo2ABAdB}.png}


\section{Producto ganancia por ancho de banda}
\label{\detokenize{sintonizados/sintonizados:Producto-ganancia-por-ancho-de-banda}}
El producto de ganancia\sphinxhyphen{}ancho de banda (designado como GBP) para un amplificador es el producto del ancho de banda del amplificador y la ganancia con la que se mide el ancho de banda.

Para los transistores, el producto de ancho de banda de ganancia de corriente se conoce como \(f_T\) o frecuencia de transición. Se calcula a partir de la ganancia de corriente de baja frecuencia (unos pocos \(kHz\)) en condiciones de prueba especificadas, y la frecuencia de corte a la cual la ganancia de corriente cae en \(-3 dB\). El producto de estos dos valores puede considerarse como la frecuencia a la que la ganancia de corriente se reduciría a 1, y la ganancia de corriente del
transistor entre la frecuencia de corte y la transición se puede estimar dividiendo \(f_T\) por la frecuencia. Por lo general, los transistores deben aplicarse a frecuencias muy por debajo de \(f_T\) para ser útiles como amplificadores y osciladores. En un transistor bipolar, la respuesta de frecuencia disminuye debido a la capacitancia interna de las uniones.
\begin{equation*}
\begin{split}GBP = |A_o| \cdot BW\end{split}
\end{equation*}
Remplazando
\begin{equation*}
\begin{split}GBP = gm\cdot R \cdot \frac{f_o}{Q_c}\end{split}
\end{equation*}
Siendo \(Q_c = R \cdot \omega_o C\)\$
\begin{equation*}
\begin{split}GBP =  \frac{gm\cdot R \cdot f_o}{R \cdot \omega_o C}\end{split}
\end{equation*}\begin{equation*}
\begin{split}GBP =  \frac{gm\cdot R \cdot f_o}{R \cdot 2 \pi f_o C}\end{split}
\end{equation*}
Simplificando
\begin{equation*}
\begin{split}GBP =  \frac{gm}{2 \pi C}\end{split}
\end{equation*}
Encontramos que el producto ganancia por ancho de banda depende de los parametros del dispositivo activo.

Cuando más pequeño sea \(C\), mayor resulta este producto, es de remarcar que el mínimo valor de \(C\) es la capacidad de salida del dispositivo activo. El producto ganancia por ancho de banda arroja una constante, así si se aumenta la ganancia se disminuye el ancho de banda y viceversa. Si se fija la ganancia el \(Q\) queda dado por esta ganancia.


\section{Amplificador multietapa sincrónico}
\label{\detokenize{sintonizados/sintonizados:Amplificador-multietapa-sincr_xf3nico}}
Se colocan en cascada n\sphinxhyphen{}etapas simples sintonizadas como las mostradas al principio de la unidad.

La transferencia de una etapa se calcula como:
\begin{equation*}
\begin{split}A_v = - g_m R \frac{1}{1 + j\chi}\end{split}
\end{equation*}
donde \(\chi(\omega) = 2Q \frac{f- f_o }{f_o}\)

Para la respuesta de n\sphinxhyphen{}etapas simple sintonizadas sincronicas
\begin{equation*}
\begin{split}A_v^n = (- g_m R \frac{1}{1 + j\chi})^n\end{split}
\end{equation*}
Donde el modulo de la transferencia
\begin{equation*}
\begin{split}|A_v|^n =   \frac{(g_m R)^n}{(\sqrt{1 + \chi^2})^n}\end{split}
\end{equation*}\begin{equation*}
\begin{split}|A_v|^n =   \frac{|\bar{A_v}_o|^n}{(1 + \chi^2)^\frac{n}{2}}\end{split}
\end{equation*}
Si se dispone de n\sphinxhyphen{}etapas simple sintonizadas sincronicas y de igual ancho de banda, a la frecuencia de \(-3dB\) de cada etapa, generada una respuesta de \(n \times -3dB\).

Para calcular el ancho de banda para n\sphinxhyphen{}etapas en cascada simple sintonizadas sincronicas, podriamos buscar las frecuencias donde la respuesta es \(\frac{-3dB}{n}\) para una unica etapa.
\begin{equation*}
\begin{split}|\bar{A_v}|^n =   \frac{1}{(1 + \chi_c^2)^\frac{n}{2}} = \frac{1}{\sqrt{2}}\end{split}
\end{equation*}\begin{equation*}
\begin{split}(1 + \chi_c^2)^\frac{n}{2} = \sqrt{2}\end{split}
\end{equation*}\begin{equation*}
\begin{split}(1 + \chi_c^2)^n = 2\end{split}
\end{equation*}\begin{equation*}
\begin{split}\chi_c = \pm \sqrt{2^\frac{1}{n} - 1}\end{split}
\end{equation*}
Entonces, calculando las frecuencias de corte inferior y superior.
\begin{equation*}
\begin{split}\chi_{ci} = 2Q \frac{f_{ci}- f_o }{f_o} = - \sqrt{2^\frac{1}{n} - 1}\end{split}
\end{equation*}\begin{equation*}
\begin{split}f_{ci} = f_o - \frac{f_o \sqrt{2^\frac{1}{n} - 1}}{2Q }\end{split}
\end{equation*}\begin{equation*}
\begin{split}\chi_{cs} = 2Q \frac{f_{cs}- f_o }{f_o} =  \sqrt{2^\frac{1}{n} - 1}\end{split}
\end{equation*}\begin{equation*}
\begin{split}f_{cs} = f_o + \frac{f_o \sqrt{2^\frac{1}{n} - 1}}{2Q }\end{split}
\end{equation*}
El ancho de banda de n\sphinxhyphen{}etapas se calcula como
\begin{equation*}
\begin{split}BW_n = f_{cs} - f_{ci} = \frac{f_o}{Q} \sqrt{2^\frac{1}{n} - 1}\end{split}
\end{equation*}

\section{Ejemplo 3}
\label{\detokenize{sintonizados/sintonizados:Ejemplo-3}}
Supongamos tres etapa simple sintonizada sincronicas en cascada a la frecuencia de \(f_o = 1MHz\), con un ancho de banda total de \(BW_3 = 100 KHz\). El elemento activo tiene una ganancia de transconductancia \(g_m = 100 mS\) y la resistencia total de la etapa es de \(R_t = 10 \Omega\).

El factor de selectividad de cada una de las etapas \(Q\)
\begin{equation*}
\begin{split}Q = \frac{f_o}{BW_3} \sqrt{2^\frac{1}{3} - 1}\end{split}
\end{equation*}\begin{equation*}
\begin{split}Q = \frac{1 MHz}{100 KHz} \sqrt{2^\frac{1}{3} - 1} = 5.098\end{split}
\end{equation*}
La transferencia de tensión en función de la frecuencia para este sistema empleando aproximación de banda angosta (en escala semilog en la frecuecia).

\sphinxincludegraphics[width=432\sphinxpxdimen,height=288\sphinxpxdimen]{{ejemplo3Av3}.png}

La transfencia en dB

\sphinxincludegraphics[width=432\sphinxpxdimen,height=288\sphinxpxdimen]{{ejemplo3Av3dB}.png}


\section{Ejemplo 4, Simple Sintonizado}
\label{\detokenize{sintonizados/sintonizados:Ejemplo-4,-Simple-Sintonizado}}
El circuito de la figura corresponde al circuito equivalente simplificado de un receptor de RF. Se desea ampificar una señal de frecuencia \(F_c = 100 MHz\).

Suponer que el inductor tiene un factor de merito de \(Q_o = 50\) y el capacitor tiene un factor de merito de \(Q_o = infinito\).

La fuente tiene una potencia disponible \(P_{disp}= 10 uW\) y su resistencia interna es \$r\_g=1K:nbsphinx\sphinxhyphen{}math:{\color{red}\bfseries{}\textasciigrave{}}Omega {\color{red}\bfseries{}\textasciigrave{}}\$.

\sphinxincludegraphics[width=508\sphinxpxdimen,height=168\sphinxpxdimen]{{ejemplo4}.png}

Donde: \(g_{11} = 1.25 mS\), \(g_{22} = 0.1 mS\) y \(g_m = 100 mS\)

Determinar para una atenuación de \(20 dB\) a \(f=120 MHz\) respecto a la frecuencia de sintonia.

Diseñar para máxima transferencia de energía a Q constante:
\begin{enumerate}
\sphinxsetlistlabels{\arabic}{enumi}{enumii}{}{.}%
\item {} 
\(R_L\)

\item {} 
\(Q_{c}\)

\item {} 
\(L\)

\item {} 
\(C\)

\item {} 
\(A_{v_o} = \frac{v_o}{v_g}\)

\item {} 
El ancho de banda \(BW\) del amplificador.

\item {} 
\(P_{in}\) 8 \(P_{R_L}\) 9 Perdidas de inserción.

\item {} 
\(|A_{v_o}|\) para \(80 MHz\), \(90 MHz\) y \(110 MHz\).

\end{enumerate}

Respuestas
\begin{enumerate}
\sphinxsetlistlabels{\arabic}{enumi}{enumii}{}{.}%
\item {} 
\(R_L\)

\end{enumerate}
\begin{equation*}
\begin{split}r_{22} = \frac{1}{1.25mS} = 10K\Omega\end{split}
\end{equation*}
Para máxima transferencia de energía a Q constante
\begin{equation*}
\begin{split}R_L = r_{22} = 5K\Omega\end{split}
\end{equation*}\begin{enumerate}
\sphinxsetlistlabels{\arabic}{enumi}{enumii}{}{.}%
\setcounter{enumi}{1}
\item {} 
\(Q_c\)

\end{enumerate}

Para el calculo de \(Q_c\) empleamos la expresión de la transferencia de tensión del simple sintonizado.
\begin{equation*}
\begin{split}|\bar{A_v}| =   \frac{1}{ \sqrt{1 + Q_c^2(\frac{f}{f_o}-\frac{ f_o}{f})^2} }\end{split}
\end{equation*}
Buscamos una atenuación de \(20 dB\) a \(f=120 MHz\).
\begin{equation*}
\begin{split}|\bar{A_v}|(120MHz) = \frac{1}{ 10^{\frac{20}{20} } }\end{split}
\end{equation*}\begin{equation*}
\begin{split}\frac{1}{10} =   \frac{1}{ \sqrt{1 + Q_c^2(\frac{120MHz}{100MHz}-\frac{ 100MHz}{120MHz})^2} }\end{split}
\end{equation*}\begin{equation*}
\begin{split}10 =    \sqrt{1 + Q_c^2(\frac{120MHz}{100MHz}-\frac{ 100MHz}{120MHz})^2}\end{split}
\end{equation*}\begin{equation*}
\begin{split}Q_c = \frac{\sqrt{10^2 - 1}}{\frac{120MHz}{100MHz}-\frac{ 100MHz}{120MHz}}\end{split}
\end{equation*}\begin{equation*}
\begin{split}Q_c = 27.136\end{split}
\end{equation*}\begin{enumerate}
\sphinxsetlistlabels{\arabic}{enumi}{enumii}{}{.}%
\setcounter{enumi}{2}
\item {} 
\(L\)

\end{enumerate}

A partir de \(Q_c\) es posible encontrar el inductor mediante la expresión
\begin{equation*}
\begin{split}\frac{1}{Q_c} =  \frac{1}{Q_o} + \frac{\omega L}{r_{ext}}\end{split}
\end{equation*}
donde \(r_{ext} = \frac{r_{22} R_L }{ r_{22}+ R_L} = 5K\Omega5\)
\begin{equation*}
\begin{split}\omega_o L = r_{ext} (\frac{1}{Q_c} -  \frac{1}{Q_o})\end{split}
\end{equation*}\begin{equation*}
\begin{split}\omega_o L = 5K\Omega (\frac{1}{27.136} -  \frac{1}{50})\end{split}
\end{equation*}\begin{equation*}
\begin{split}\omega_o L =  12.48 \Omega\end{split}
\end{equation*}
El inductor \(L = 19.86 nHy\)
\begin{enumerate}
\sphinxsetlistlabels{\arabic}{enumi}{enumii}{}{.}%
\setcounter{enumi}{3}
\item {} 
\(C\)

\end{enumerate}

El capacitor \(C\)
\begin{equation*}
\begin{split}C = \frac{1}{L \omega_o^2}\end{split}
\end{equation*}\begin{equation*}
\begin{split}C = 127.50 pF\end{split}
\end{equation*}\begin{enumerate}
\sphinxsetlistlabels{\arabic}{enumi}{enumii}{}{.}%
\setcounter{enumi}{4}
\item {} 
\(|A_{vo}|\)

\end{enumerate}
\begin{equation*}
\begin{split}|A_{vo}| = \frac{r_{11}}{r_{11}+r_g} \times gm R_t\end{split}
\end{equation*}\begin{equation*}
\begin{split}|A_{vo}| = \frac{r_{11}}{r_{11}+r_g} \times gm Q_c \omega_o L\end{split}
\end{equation*}\begin{equation*}
\begin{split}|A_{vo}| = \frac{800\Omega}{800\Omega+1K\Omega} \times 0.1mS 27.13 2 \pi 100MHz 19.86 nHy\end{split}
\end{equation*}\begin{equation*}
\begin{split}|A_{vo}| =  15.05\end{split}
\end{equation*}\begin{enumerate}
\sphinxsetlistlabels{\arabic}{enumi}{enumii}{}{.}%
\setcounter{enumi}{5}
\item {} 
\(BW\)

\end{enumerate}
\begin{equation*}
\begin{split}BW = \frac{f_o}{Q_c} =  \frac{100 MHz}{27.13} =  3.685 MHz\end{split}
\end{equation*}\begin{enumerate}
\sphinxsetlistlabels{\arabic}{enumi}{enumii}{}{.}%
\setcounter{enumi}{6}
\item {} 
\(P_{in}\)

\end{enumerate}


\section{Ganacia de potencia}
\label{\detokenize{sintonizados/sintonizados:Ganacia-de-potencia}}
La expresión general para ganancia de potencia es:
\begin{equation*}
\begin{split}G = \frac{ |y_{21}|^2 \cdot \Re(y_{L}) }{ |y_{22} + y_{L}|^2  \cdot  \Re(y_{11} - \frac{ y_{12}\cdot y_{21}}{y_{22} + y_{L} })}\end{split}
\end{equation*}
La ecuación se aplica a circuitos sin retroalimentación externa.

También se puede usar con circuitos que tienen retroalimentación externa si los parámetros compuestos y del transistor y la red de retroalimentación son sustituidos por los parámetros del transistor y en la ecuación. Los parámetros compuestos y se determinan considerando que el transistor y la red de retroalimentación son dos “cajas negras” en paralelo:
\begin{equation*}
\begin{split}y_{11c} = y_{11t} + y_{11f}\end{split}
\end{equation*}\begin{equation*}
\begin{split}y_{12c} = y_{12t} + y_{12f}\end{split}
\end{equation*}\begin{equation*}
\begin{split}y_{21c} = y_{21t} + y_{21f}\end{split}
\end{equation*}\begin{equation*}
\begin{split}y_{22c} = y_{22t} + y_{22f}\end{split}
\end{equation*}
Donde: \(y_{11c}\), \(y_{12c}\), \(y_{21c}\), \(y_{22c}\) son los parámetros compuestos y de la combinación paralela de transistor y red de retroalimentación.

\(y_{11t}\), \(y_{12t}\), \(y_{21t}\), \(y_{22t}\) son los parámetros y del transistor.

\(y_{11f}\), \(y_{12f}\), \(y_{21f}\), \(y_{22f}\) son los parámetros y de la red de retroalimentación.

Tenga en cuenta que, dado que este enfoque trata la combinación del transistor y la red de retroalimentación como una única “caja negra” con \(y_{11c}\), \(y_{12c}\), \(y_{21c}\), \(y_{22c}\) como sus parámetros y, los parámetros compuestos y pueden ser sustituidos, en cualquiera de las ecuaciones de diseño aplicables a un análisis lineal activo de dos puertos.

Los amplificadores neutralizados y unilateralizados son casos especiales de este concepto general, y las ecuaciones asociadas con esos casos especiales se darán más adelante.

La ecuación proporciona una solución para la ganancia de potencia de la red activa lineal (transistor) solamente. Las redes de entrada y salida se consideran parte de la fuente y la carga, respectivamente. Por lo tanto, deben tenerse en cuenta dos puntos importantes:
\begin{enumerate}
\sphinxsetlistlabels{\arabic}{enumi}{enumii}{}{.}%
\item {} 
La ganancia de potencia calculada a partir de la ecuación de \(G\) no tendrá en cuenta las pérdidas de red. La pérdida de la red de entrada reduce la potencia entregada al transistor. La potencia perdida en la red de salida se calcula como salida de potencia útil, ya que la admitancia de carga \(YL\) es la combinación de la red de salida y su carga.

\item {} 
La ganancia de potencia es independiente de la fuente admitida. Una falta de coincidencia de entrada da como resultado que se entregue menos potencia de entrada al transistor. En consecuencia, tenga en cuenta que la ecuación de \(G\) no contiene el término \(Ys\).

\end{enumerate}

La ganancia de potencia de un transistor junto con sus redes de entrada y salida asociadas se puede calcular midiendo las pérdidas de la red de entrada y salida, y restándolas de la ganancia de potencia calculada con la ecuación de \(G\).

En algunos casos, puede ser conveniente incluir los efectos de adaptación de entrada en cálculos de ganancia de potencia. Un término conveniente es ganancia de transductor \(G_T\), definida como potencia de salida entregada a una carga por el transistor, dividida por la potencia de entrada máxima disponible desde la fuente.

La ecuación para la ganancia del transductor es:
\begin{equation*}
\begin{split}G_T = \frac{4 \cdot |y_{21}|^2 \cdot \Re(Y_{s}) \cdot \Re(Y_{L}) }{ |(y_{11} + y_{s})  \cdot (y_{22} + y_{L}) - (y_{12} + y_{21})|^2}\end{split}
\end{equation*}
En esta ecuación, \(YL\) es la admitancia de carga de transistor compuesta, compuesta tanto de la red de salida como de su carga, e Ys es la admitancia de fuente de transistor compuesta, compuesta por la red de entrada y su fuente. Por lo tanto, la ganancia del transductor incluye los efectos del grado de coincidencia de admitancia en los terminales de entrada del transistor, pero no tiene en cuenta las pérdidas de la red de entrada y salida. Como en la ecuación de \(G\), los parámetros
y compuestos de una combinación de red de retroalimentación de transistor pueden ser sustituidos por los parámetros y del transistor cuando se usa dicha combinación. La ganancia máxima disponible (MAG) es una figura de mérito de transistor de uso frecuente.

El MAG es la ganancia de potencia teórica de un transistor con su admitancia de transferencia inversa \(y_{12}\) igual a cero, y sus admitancias de fuente y carga coinciden de forma conjugada con \(y_{12}\) e \(y_{22}\), respectivamente.

Si \(y_{12} = 0\), el transistor exhibe una admitancia de entrada igual a \(y_{11}\) y una admitancia de salida igual a \(y_{22}\).

La ecuación para MAG, por lo tanto, se obtiene resolviendo la expresión de ganancia de potencia general, ecuación de \(G\), con las condiciones:
\begin{equation*}
\begin{split}y_{12} = 0\end{split}
\end{equation*}\begin{equation*}
\begin{split}y_{s}  = y_{11}*\end{split}
\end{equation*}\begin{equation*}
\begin{split}y_{L}  = y_{22}*\end{split}
\end{equation*}
donde \(*\) denota conjugado, lo que produce:
\begin{equation*}
\begin{split}MAG = \frac{ |y_{21}|^2  }{ 4 \cdot \Re(y_{11}) \cdot \Re(y_{22})}\end{split}
\end{equation*}
MAG es una figura de mérito solamente, ya que es físicamente imposible reducir \(y_{12} = 0\), sin cambiar los otros parámetros del transistor. Se puede usar una red de retroalimentación externa para lograr un compuesto \(y_{12}\) de cero, pero luego los otros parámetros compuestos también se modificarán de acuerdo con las relaciones dadas en la discusión del transistor compuesto \sphinxhyphen{} red de retroalimentación “caja negra”.


\section{\protect\(G_{max}\protect\)}
\label{\detokenize{sintonizados/sintonizados:G__max_}}
\(G_{max}\), la ganancia de transductor más alta posible sin retroalimentación externa, forma un caso especial del amplificador sin retroalimentación. Las admisiones de fuente y carga requeridas para lograr \(G_{max}\) pueden calcularse a partir de lo siguiente:
\begin{equation*}
\begin{split}g_{s}  = \frac{1}{2 \cdot  \Re(y_{22})} \cdot  \sqrt{ [2 \cdot \Re(y_{11}) \cdot \Re(y_{22}) - \Re(y_{12} \cdot y_{21})]^2 - |y_{12} \cdot y_{21}|^2 }\end{split}
\end{equation*}\begin{equation*}
\begin{split}b_{s}  = - \Im(y_{11}) + \frac{\Im(y_{12} \cdot y_{21})}{2\cdot  \Re(y_{22})}\end{split}
\end{equation*}\begin{equation*}
\begin{split}g_{l}  = \frac{1}{2 \cdot  \Re(y_{11})} \cdot  \sqrt{ [2 \cdot \Re(y_{11}) \cdot \Re(y_{22}) - \Re(y_{12} \cdot y_{21})]^2 - |y_{12} \cdot y_{21}|^2 }\end{split}
\end{equation*}\begin{equation*}
\begin{split}b_{s}  = - \Im(y_{22}) + \frac{\Im(y_{12} \cdot y_{21})}{2\cdot  \Re(y_{11})}\end{split}
\end{equation*}
La magnitud de \(Gmax\) puede calcularse a partir de las siguientes expresiones:
\begin{equation*}
\begin{split}G_{max} = \frac{|y_{21}|^2}{ [2 \cdot \Re(y_{11}) \cdot \Re(y_{22}) - \Re(y_{12} \cdot y_{21})] + \sqrt{ [2 \cdot \Re(y_{11}) \cdot \Re(y_{22}) - \Re(y_{12} \cdot y_{21})]^2 - |y_{12} \cdot y_{21}|^2 }  }\end{split}
\end{equation*}
{
\sphinxsetup{VerbatimColor={named}{nbsphinx-code-bg}}
\sphinxsetup{VerbatimBorderColor={named}{nbsphinx-code-border}}
\begin{sphinxVerbatim}[commandchars=\\\{\}]
\llap{\color{nbsphinxin}[ ]:\,\hspace{\fboxrule}\hspace{\fboxsep}}
\end{sphinxVerbatim}
}

{
\sphinxsetup{VerbatimColor={named}{nbsphinx-code-bg}}
\sphinxsetup{VerbatimBorderColor={named}{nbsphinx-code-border}}
\begin{sphinxVerbatim}[commandchars=\\\{\}]
\llap{\color{nbsphinxin}[ ]:\,\hspace{\fboxrule}\hspace{\fboxsep}}
\end{sphinxVerbatim}
}

{
\sphinxsetup{VerbatimColor={named}{nbsphinx-code-bg}}
\sphinxsetup{VerbatimBorderColor={named}{nbsphinx-code-border}}
\begin{sphinxVerbatim}[commandchars=\\\{\}]
\llap{\color{nbsphinxin}[ ]:\,\hspace{\fboxrule}\hspace{\fboxsep}}
\end{sphinxVerbatim}
}



\renewcommand{\indexname}{Índice}
\printindex
\end{document}